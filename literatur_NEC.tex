Hello Frauke, da ich nicht weiß, wie ich mit der Literatur verfahren soll, kopiere ich sie hierhin. Da die Textteile, die ich vorbereitet hatte, erheblich länger sind, als das, was ich hier hin kopiert habe, könnten manche Angaben wieder rausfallen. Das kann ich gern am Ende unseres Prozesses kontrollieren: BMWi (2014): Die Energie der Zukunft. Erster Fortschrittsbericht zur Energiewende, Berlin: BMWi. Brand, Ulrich und Wissen, Markus (2017): Imperiale Lebensweise. Zur Ausbeutung von Mensch und Natur im globalen Kapitalismus, München: oekom. Brischke, Lars-Arvid, Leuser, Leon, Duscha, Markus, et al. (2016): Energiesuffizienz – Strategien und Instrumente für eine technische, systemische und kulturelle Transformation zur nachhaltigen Begren- zung des Energiebedarfs im Konsumfeld Bauen / Wohnen. Endbericht, Heidelberg: ifeu. Christ, Michaela (2015): „Künstliche Beleuchtung, Handlungsspielräume und wirtschaftliches Wachstum“, in: Appelt, Siegrun (Hrsg.), Langsames Licht / Slow Light. Lichtprojekte in der Wachau. Dortmund: Kettler, S. 139–145. Christ, Michaela (2016): „Die Zukunft liegt im Dunkeln. Dynamiken von Wachstum und künstlicher Beleuchtung in der Moderne“, in psychosozial 143 (Schwerpunktthema Postwachstum – Subjektivität – Demokratie, herausgegeben von Oliver Decker und Dennis Eversberg), 39, 1, S. 11–24. David, Martin und Schönborn, Sophia (2016): Die Energiewende als Bottom-up-Innovation. Wie Pionierprojekte das Energiesystem verändern, München: oekom. Diamond, Jarred (2004): Collapse: How Societies Choose to Fail or Succeed, New York: Viking Press. Fischer, Corinna und Grießhammer, Rainer (2013): Mehr als nur weniger. Suffizienz: Begriff, Begründung und Potenziale (Öko-Insititut Working Paper 2/2013), Freiburg: Öko-Institut. Fischer-Kowalski, Marina, Mayer, Andreas und Schaffartzik, Anke (2011): „Zur sozialmetabolischen Transformation von Gesellschaft und Soziologie“, in: Groß, Matthias (Hrsg.), Handbuch Umweltsoziologie. Wiesbaden: VS, S. 97–120. Haberl, Helmut, Fischer-Kowalski, Marina, Krausmann, Fridolin, et al. (2011): „A socio-metabolic transition towards sustainability? Challenges for another Great Transformation“, in Sustainable Development, 19, 1, S. 1–14. Haberl, Helmut, Hüttler, Walter und Fischer-Kowalski, Marina (1997): Gesellschaftlicher Stoffwechsel und Kolonisierung von Natur. Ein Versuch in sozialer Ökologie, Amsterdam: G+B Verlag Fakultas. Huber, Joseph (1995): „Nachhaltige Entwicklung durch Suffizienz, Effizienz und Konsistenz“, in: Fritz, Peter (Hrsg.), Nachhaltigkeit in naturwissenschaftlicher und sozialwissenschaftlicher Perspektive. Stuttgart: Wissenschaftliche Verlagsgesellschaft, S. 31–46. Kosow, Hannah und Gaßner, Robert (2008): Methoden der Zukunfts- und Szenarioanalyse. Überblick, Bewertung und Auswahlkriterien, Werkstattbericht 103, Berlin: Institut für Zukunftsstudien und Technologiebewertung. Krausmann, Fridolin und Fischer-Kowalski, Marina (2010): „Gesellschaftliche Naturverhältnisse. Globale Transformationen der Energie- und Materialflüsse“, in: Sieder, Reinhard und Langthaler, Ernst (Hrsg.), Globalgeschichte 1800-2010. Wien: Böhlau, S. 39–66. Lessenich, Stephan (2016): Neben uns die Sintflut. Die Externalisierungsgesellschaft und ihr Preis, Berlin: Hanser. McNeill, John R. (2014): The Great Acceleration: An Environmental History of the Anthropocene since 1945, Cambridge: Harvard University Press. Öko-Institut (2017): Heute. Morgen. Zukunft. Visionen und Wege für eine nachhaltige Gesellschaft, Freiburg: Öko-Institut. Pfister, Christian, Kaufmann-Hayou, Ruth, Messerli, Paul, et al. (1995): „Das "1950er Syndrom". Zusammenfassung und Synthese“, in: Pfister, Christian (Hrsg.), Das 1950er Syndrom. Der Weg in die Konsumgesellschaft. Bern, Stuttgart, Wien: Paul Haupt, S. 21–47. Robinson, John (2003): „Future subjunctive: backcasting as social learning“, in Futures, 35, 8, S. 839–856. Robinson, John, Burch, Sarah, Talwar, Sonia, et al. (2011): „Envisioning sustainability: Recent progress in the use of participatory backcasting approaches for sustainability research“, in Technological Forecasting and Social Change, 78, 5, S. 756–768. Rosa, Hartmut (2005): Beschleunigung, Frankfurt/M.: Suhrkamp. Schneidewind, Uwe und Zahrnt, Angelika (2013): Damit gutes Leben einfacher wird. Perspektiven einer Suffizienzpolitik, München: oekom. Sieferle, Rolf Peter (1982) Der unterirdische Wald. München: Beck. Sommer, Bernd (2011): „Interdependenzen und Ungleichzeitigkeiten im Kontext des anthropogenen Klimawandels“, in Leviathan – Berliner Zeitschrift für Sozialwissenschaft, 39, 1, S. 55–71. Sommer, Bernd (2016): „Ist eine nachhaltige Moderne möglich? Zum Verhältnis von Wachstum, sozialer Differenzierung und Naturverbrauch“, in: AK Postwachstum (Hrsg.), Wachstum - Krise und Kritik. Die Grenzen der kapitalistisch-industriellen Lebensweise. Frankfurt am Main / New York: Campus, S. 269-288. Sommer, Bernd, Kny, Josefa, Stumpf, Klara, et al. (2016): „Gemeinwohl-Ökonomie: Baustein zu einer ressourcenleichteren Gesellschaft?“, in: Rogall, H. (Hrsg.), Im Brennpunkt Ressourcenwende - Transformation zu einer ressourcenleichten Gesellschaft. Jahrbuch Nachhaltige Ökonomie, Bd. 5. Marburg, S. Sommer, Bernd und Schad, Miriam (2014): „Climate change and society: possible impacts and prospective developments“, in Meteorologische Zeitschrift, 24, 2, S. 137–145. Sommer, Bernd und Welzer, Harald (2014): Transformationsdesign. Wege in eine zukunftsfähige Moderne, München: oekom. Steffen, Will, Crutzen, Paul J. und McNeill, John R. (2007): „The Anthropocene: Are Humans Now Overwhelming the Great Forces of Nature?“, in Ambio, 36, 8, S. 614–621. Stumpf, K.H., Baumgärtner, S., Becker, C.U. , et al. (2015): „The justice dimension of sustainability. A systematic and general conceptual framework“, in Sustainability, 7, 6, S. 7438-7472. Wallerstein, Immanuel, Collins, Randall, Mann, Michael, et al. (2014) (Hrsg.): Stirbt der Kapitalismus? Fünf Szenarien für das 21. Jahrhundert. Frankfurt am Main: Campus-Verlag., Frankfurt am Main: Campus. WBCSD (1997): Explorig Sustainable Development. World Business Council for Sustainable Development. Global Scenarios 2000-2050. Summary Brochure, Genf: WBCSD. WBGU (2011): Welt im Wandel: Gesellschaftsvertrag für eine Große Transformation, Berlin: WBGU. Welzer, Harald (2011) Mentale Infrastrukturen. Wie das Wachstum in die Welt und in die Seelen kam. Berlin: Heinrich Böll Stiftung, Band 14.


ERGÄNZUNG KLARA:
Ergänzungen: Kny, Josefa/Schmies, Maximilian/Sommer, Bernd/Welzer, Harald/Wiefek, Jasmin: Von der Nische in den Mainstream: Wie gute Beispiele nachhaltigen Handelns in einem breiten gesellschaftlichen Kontext verankert werden. Umweltbundesamt - TEXTE 86/2015. Dessau-Roßlau.