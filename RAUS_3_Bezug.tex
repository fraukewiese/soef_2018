\section{Bezug zur Sozial-ökologischen Forschung und zu den Förderzielen}

% Bezug zu Sozial-ökologischer Forschung
% inderdisziplinär
Erster Bezugspunkt zur sozial-ökologischen Forschung ist die Interdisziplinarität des beabsichtigten Nachwuchsforschungsverbundes. Sie soll insbesondere über die Zusammenarbeit von Technik- und Sozialwissenschaftler*innen erfolgen (zur disziplinären Zusammensetzung siehe \ref{sec:4}). Mit der Weiterenwicklung interdisziplinärer Szenario-Methoden trägt die Nachwuchsforschungsgruppe zur Integration disziplinären Wissens für Energiewende und Klimaschutz bei.

% Personennennung inter
%Benjamin Best bringt hierfür als interdisziplinärer ausgebildeter Soziologe mit technischer und sozialwissenschaftlicher Expertise hervorragende Voraussetzungen für die Ko-Leitung mit. Die sozialwissenschaftliche Seite wird durch ein volkswirtschaftliches und ein politikwissenschaftliches Promotionsvorhaben (Politische Ökonomie, Internationale Politik, Politische Ökologie) gestärkt. Die Wirtschaftsingenieurin Frauke Wiese bringt ihre langjährige Erfahrung in der Energiesystem-Modellierung und den technisch-ökonomischen Blickwinkel auf die Herausforderungen der Energiewende ein. 

% transdisziplinär
Zweiter Bezugspunkt ist der transdisziplinäre Ansatz, eng mit Praxispartner*innen zu kooperieren und hier auf den qualitativen und innovativen methodischen Ansatz der „transdisziplinären Erzählungen“ (REFERENZ Biesecker, Breitenbach Winterfeld 2016) zurückgreifen zu können. Diese sind auch für die lebensweltliche Einbettung des Vorhabens besonders geeignet. Im Projekt ist eine enge Zusammenarbeit mit kommunalen Praxispartnern bei der Szenarienerstellung und -bewertung vorgesehen, wobei auf bestehende Kooperatione aufgebaut werden kann (s. \ref{sec:5}).

%Eine Schwachstelle sehen wir in der immer noch vorhandenen Tendenz, dass entweder Wissenschaft von der Praxis absieht bzw. diese instrumentalisiert (gerade im Modell wird erst von der Wirklichkeit abstrahiert, um anschließend all das an Wirklichkeit auszuschließen, was nicht zum Modell passt) oder die Praxis der Wissenschaft – insbesondere in der Auftragsforschung – ihre Ergebnisse vorschreibt. Für die Entwicklung von Alternativen hierzu ist Benjamin Best als Ko-Leiter durch seine Arbeiten in den Projekten Innovation City (Bottrop) und im sozial-ökologischen Forschungsverbund DoNaPart (\url{https://projekt-donapart.de/}) gut ausgestattet. 

%In diesem Verbundprojekt wird ein Lebensweltbezug hergestellt, indem die Wissenschaftler*innen in einem Stadtteil einen performativen Partizipationsprozess durchführen und dessen Empowermentwirkungen durch ein Prä- Post-Design analysieren. 
% international und gender
Weiter und in Bezug zu den Förderzielen soll sowohl die internationale Dimension (z.B. Auswirkungen deutscher Energiewendepolitiken auf andere Länder) als auch der über die Integration der Genderdimension mögliche Perspektivwechsel in den Promotionsvorhaben bearbeitet werden.

% Inhaltlicher Bezugspunkt zu SOEF
Theamtisch widmet sich die Nachwuchsforschungsgruppe einer in der Sozial-ökologischen Forschung als zentral geltenden Nachhaltigkeits-Herausforderun: Der Umwelt- und gesellschaftsverträglichen Transformation des Energiesystems. Mit der Fokussierung auf Suffzienz nehmen die Nachwuchswissenschaftler*innen Bezug zu Ressourcenschonung sowie dem Sozial-ökologischen Forschungs-Schwerpunkt Nachhaltiges Wirtschaften.

% Bezug zu Förderzielen: Institutionell und Kooperation außeruniversitär
Die Weiterentwicklung institutioneller Kapazitäten zur Durchführung transdisziplinärer Nachhaltigkeitsforschung wird durch die Verankerung der Nachwuchsgruppe am Interdisziplinären Institut für Umwelt-, Sozial und Humanwissenschaften an der Europa-Universität Flensburg sowie am Wuppertal Institut geschaffen. Der Wissensaustausch zwischen Hochschule und außeruniversitärer Forschungseinrichtung wird verstärkt durch die gemeinsame Leitung von Wissenschaftler*innen beider Institutionen. In der eigenverantwortlichen Arbeitsgruppe bekommen junge Wissenschaftler*innen die Möglichkeit sich auf der Basis ihres disziplinären Vorwissens Sozial-ökologischen Fragestellungen zu widmen. Möglichkeiten zur Weiterqualifizierung werden durch die Nachwuchsgruppe für wissenschaftliche Mitarbeiter*innen geschaffen, die zwar schon jetzt an den Schnittstellen forschen, dafür aber nicht den entsprechenden Rahmen haben um sich auch wissenschaftlich zu qualifizieren.