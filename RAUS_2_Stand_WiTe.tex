\section{Stand von Wissenschaft und Technik sowie eigene Vorarbeiten}
\label{sec:2}

% gemeinsamer Anfang 
Nachhaltigkeitsforschung, sozial-ökologische Forschung und Transformationsforschung sind seit langem mit Fragen der Umsetzung befasst: Wie ist es möglich, vom Wissen zum Handeln zu kommen ? \cite{BMBF2008} Wie kann geschehen, was geschehen muss ? \cite{Linz2000} Und wie können Klimaschutzziele, wie kann ein deutlich geringerer Naturverbrauch erreicht werden, ohne zivilisatorische Errungenschaften zu gefährden ? \cite{Sommer2016,WGBU2011} Die Antworten fallen unterschiedlich aus und die Bearbeitung dieser Frage geschieht in unterschiedlicher Weise. So hat der Nachhaltigkeitspfad der Suffizienz lange im Schatten von Effizienz (mehr Wohlstand bei geringerem Ressourcenverbrauch) und von Konsistenz (Wechsel der Stoffbasis) gestanden (REFERENZ Winterfeld).

%Die Transformationsforschung hat sich zunächst stark am von der System- und Managementtheorie kommenden Transition-Ansatz orientiert (REFERENZ Rotmans, Geels 2004; \cite{Loorbach2010}). Innerhalb der Debatten haben Szenarien und Modelle zum einen stets die Funktion gehabt, Orientierungswissen zu generieren um zu zeigen, dass beispielsweise eine Energiewende (Ausstieg aus der Atomenergie und Umstieg auf erneuerbare Energieträger) möglich ist. Zugleich werden sie allerdings wegen ihres Abstraktionsgrades, der begrenzten Abbildung von Teilaspekten und -systemen unter Vernachlässigung lebensweltlich relevanter Aspekte und dadurch begrenzter Aussagekraft und Realitätsnähe immer wieder auch kritisiert. Hierzu gehört insbesondere auch die fehlende Abbildung der Nachfrageseite bzw. von Suffizienz in Energie- und Klimaschutzszenarien. 
%Das wurde von \cite{SAMADI2017} festgestellt und in dem aktuell am Umweltbundesamt laufenden Projekt „Energieverbrauchsreduktion durch Verhaltensänderung“ bestätigt, wie auf dem Fachgespräch am 19.03.2018 in Dessau präsentiert. Zudem werden Verlagerungseffekte und Externalitäten von Energie- und Klimaschutzentwicklungen in der Regel nicht diskutiert. Hier fehlt es an der Berücksichtigung der Zusammenhänge zwischen Energie- auf Ressourcenfragen ebenso wie die geografische Verlagerung von Verbräuchen und Emissionen im globalen Kontext.
%Für die hier skizzierte Nachwuchsgruppe sind neben den oben genannten sowie den eigenen Vorarbeiten das Konzept des Umweltbundesamtes zur absoluten Verminderung des Energiebedarfs \cite{UBA2016} sowie die Arbeiten von Laura Spengler zu Suffizienz-Politiken relevant \cite{Spengler2016,Spengler2018}.

\subsection*{Energiesystem-Analyse und Modellierung}
% ESM wichtig in Politik-Beratung
Trotz existierender Kritik, nehmen Energiesystem-Modelle eine wichtige Rolle in Wissenschaft und politischer Beratung zu Klima- und Energiepolitik ein \cite{Dieckhoff2015}.

% Entwicklung
Wurde Energiesystem-Modellierung in ihrer Anfangszeit in den 1950ern nur zur Planung neuer Kapazitäten genutzt \cite{Kagiannas2004}, nahm die Bedeutung für die Planung von nachhaltigeren Energiequellen nach der Ölkrise in den Siebzigern zu \cite{Wei2006}. Ein wichtiger Beitrag der Modelle war es dann, die Realisierbarkeit von 100\% Erneuerbare Energien Strom-Systemen darzulegen, sowohl für Deutschland \cite{SRU2011} als auch für Europa \cite{Hohmeyer2015}. Inzwischen geht es nicht mehr um das OB, sondern eher um das WIE der Energiewende. Modelle tragen u.a. zu Fragen der Marktregeln, optimalem Mix von Wind- und Solarenergie, Netz- und Speicherbedarf bei.  

% Komplexität erhöht sich
Da die Bedeutung der Sektorkopplung zunimmt \cite{Quaschning2016} hat sich die integrierte Betrachtung der Sektoren Strom, Wärme, Mobilität inzwischen etabliert. Gleichzeitig nimmt die zeitliche und räumliche Auflösung der Modelle zu, um Effekte von fluktuierender Einspeisung von Wind und Solar besser abbilden zu können. Genauso wie die Anzahl hat auch die Komplexität der Modelle in den letzten Jahren stark zugenommen.

% Transparenz wichtig für Partizipation wichtig
In den letzten Jahren wurden die Modelle verstärkt für ihren Black-Box-Charakter kritisiert \cite{Pfenninger2017, Pfenninger2017b,Cao2016}. \cite{Wiese2015} formuliert den Bedarf an Transparenz in der Energiesystem-Modellierung um Partizipation zu ermöglichen \cite{Wiese2014} als Grundlage für Transparenz, die essentiell für eine partizipative Energiewende ist.

% Interdisziplinär wichtig, Sozialwissenschaft unterrepräsentiert
Außerdem wird zunehmend ein Bedarf an interdisziplinären Ansätzen in der Energiesystem-Modellierung identifiziert \cite{Wiese2018,Pfenninger2014,Schuitema2017}. Die Sozialwissenschaft ist unterrepräsentiert in Energieforschung \cite{Sovacool2014}, was in Anbetracht der Relevanz der Sozialwissenschaften für eine gesellschaftliche Energiewende \cite{Sovacool2015} verwunderlich ist.

% Sozialwissenschaftliche Ansätze bisher
Ein Bereich, in dem sozialwissenschatliche Ansätze in der Energiewende-Modellierung bereits zur Anwendung kommen, ist die Forschung zu Akzeptanz von neuen Netzleitungen und Windkraftanlagen \cite{vernetzen2016} und dem Ausstieg aus der Kohlekraftwerken \cite{Heinrichs2017}.

% Gesellschaftliche Einbettung Szenarien
Um Modell-basierte Szenarien in einen gesellschaftlichen Kontext einzubetten, wenden \cite{WEIMERJEHLE2016} Kontext-Szenarien \cite{WEIMERJEHLE2006} an. Es gibt also erste Ansätze in  sozio-technischen Energieszenarien qualitative und quantitative Elemente der Systemanalyse zu verknüpfen.

% Suffizienz in der Energiesystem-Modellierung quantifiziert?
Trotzdem gibt es noch wenige Energieszenarien, die Suffizienz in die Analyse aufnehmen. Eine Ausnahme ist das französische Negawatt-Szenario, das Suffizienz neben Konsistenz und Effizienz abbildet \cite{negawatt2017} und mit halbiertem Energieverbrauch bei bleibendem hohen Standard an Energiedienstleistungen kalkuliert. Eine Analyse verschiedener Energie-Szenarien zeigt jedoch, dass Suffizienz trotz des potentiellen Beitrags kaum untersucht wird \cite{SAMADI2017}.

% WEITERE NOTIZEN
% Wichtig andere Aspekte miteinzubringen, Perspektive der Resilienz als verknüpfendes Element \cite{Wiese2016}.
% Es gibt immer implizite Annahmen in Energieszenarien
% (Johannes): Energieszenarien für DE (HERON-Projekt, LEAP)

\subsection*{Sozial-ökologische Transformationsforschung}
In der interdisziplinären Nachhaltigkeitsforschung ist in den vergangenen Jahren ein neuer Forschungszweig entstanden, der sich mit der gesellschaftlichen Transformation in Richtung Nachhaltigkeit befasst. Gesellschaftliche Transformationsprozesse werden hier als zukunftsorientierte Gestaltungsaufgabe verstanden (Sommer und Welzer 2014). Dies gilt auch für die Transformation des Energieregimes. Dass die Etablierung einer klimafreundlichen und nachhaltigen Energieversorgung grundsätzlich technisch machbar und ökonomisch zu bewältigen wäre, gilt als sehr wahrscheinlich. Dafür, dass dies trotz der Virulenz der Klimakrise nicht (bzw. nicht in der gebotenen Geschwindigkeit) geschieht, werden vor allem gesellschaftliche Barrieren und Zwänge ausgemacht. Im Zentrum des Interesses stehen daher gesellschaftliche Fragen einer solchen Nachhaltigkeitstransformation, wie Governance-Aspekte sowie nachhaltige soziale Praktiken und die Bedingungen ihrer Diffusion.
Die Geschichte der Energieproduktion und der Energienutzung ist eine der permanenten Effizienzsteigerungen und zugleich eine des sukzessiven Verlustes nachhaltiger, damit auch suffizienter gesellschaftlicher Reproduktion und entsprechender sozialer Praktiken. Historisch betrachtet lässt sich konstatieren: Die energetische Dichte der verwendeten Rohstoffe nahm in den vergangen rund 300 Jahren mit jeder neuen Ressource zu (von Biomasse über Kohle zu Erdöl/Erdgas und atomare Energie). Zusätzlich ließen technische Innovationen die Ressourcenproduktivität steigen. Beides trug, vermittelt über sich verändernde soziale Ordnungen, Lebensweisen und damit einhergehenden Normen- und Wertewandel, zu einer Steigerung des absoluten Ressourcen- und Naturverbrauchs bei (Krausmann und Fischer-Kowalski 2010). Heute sind diese Verbräuche auf einem historisch einzigartig hohen Niveau – mit gravierenden, zum Teil irreversiblen Folgen für Gesellschaften und Natur.
Vor dem Hintergrund der sozial-ökologischen Auswirkungen dieses hohen Naturverbrauchs, der Zerstörung von Ökosystemen, der zunehmenden Übernutzung ökologischer Senken sowie den unkalkulierbaren sozialen und ökologischen Risiken, die mit der Nutzung atomarer Energie verbunden sind, ist in Deutschland die Energiewende auf den Weg gebracht worden; vorbereitet, befördert und ermöglicht durch Jahrzehnte der umweltpolitischen und zivil-gesellschaftlichen Auseinandersetzungen einerseits und die Arbeit von Pionier*innen der regenerativen Energieversorgung andererseits (David und Schönborn 2016). Bei der Umsetzung der Energiewende kommen bisher vor allem technische Maßnahmen zum Einsatz. „Die Energiewende basiert auf zwei Säulen: erneuerbare Energien und Energieeffizienz“, heißt es entsprechend im Ersten Fortschrittsbericht zur Energiewende des Bundesministeriums für Wirtschaft und Energie (BMWi 2014: 5). Energiesuffizienz hingegen, das heißt die Begrenzung und langfristige Reduktion des absoluten Energieverbrauchs durch soziale Innovationen, Exnovationen und verändertes Nutzungsverhalten, spielt bislang noch eine untergeordnete Rolle – obwohl die Reflektion der bis dato ergriffenen Maßnahmen zeigt, dass diese zur Senkung des absoluten Verbrauchs nicht ausreichen werden (Brischke et al. 2016). Ohne Suffizienz wird die Energiewende, das wird in der sozial-ökologischen Forschung seit längerer Zeit thematisiert, nicht zu schaffen sein (Huber 1995; Fischer und Grießhammer 2013; Schneidewind und Zahrnt 2013).

Um zu verstehen, wie Suffizienz, suffiziente Lebensweisen und soziale Praktiken eine breitere gesellschaftliche Basis erreichen können, scheint ein Blick in die Vergangenheit als auch in die Zukunft sinnvoll und notwendig.

Das Norbert Elias Center for Transformation Design & Research (NEC) der EUF ist ein sozialwissenschaftliches Forschungszentrum zur Untersuchung gesellschaftlicher Veränderungsprozesse, insbesondere in Hinblick auf Fragen der Nachhaltigkeit und Zukunftsfähigkeit zeitgenössischer Gesellschaften. Am Center werden historische Prozesse umfassenden gesellschaftlichen Wandels rekonstruiert, um aus den gewonnenen Erkenntnissen Handlungsoptionen, mögliche Entwicklungslinien und Gestaltungsszenarien für zeitgenössische und künftige sozial-ökologische Wandlungsprozesse abzuleiten (Christ 2015, 2016), als auch gegenwartsorientierte Fragestellungen zum Zusammenhang von Klima, Kultur und Nachhaltigkeit bearbeitet (Sommer 2011; Sommer und Schad 2014; Stumpf et al. 2015; Kny et al. 2015, Sommer 2016).
Als Strategie die auf veränderte Handlungsmuster und Nutzungsaspekte zielt, steht Suffizienz im Fokus des BMBF geförderten Forschungsprojekts „Entwicklungschancen und -hemmnisse einer suffizienzorientierten Stadtentwicklung (EHSS)“ (2017-2020). In diesem Forschungsprojekt werden in vergleichender Perspektive und im Rahmen eines Reallabors auf kommunaler Ebene Erfolgsbedingungen und Barrieren suffizienzorientierter Stadtentwicklung untersucht. Eine Suffizienzpolitik in der Stadtentwicklung schafft Strukturen, die suffizienzorientierte (Alltags-) Praktiken erleichtern – etwa im Bereich der Mobilität, des Wohnens oder der Ernährung.
Auch im Rahmen des ebenfalls vom BMBF geförderten Forschungsprojekts „Gemeinwohl-Ökonomie im Vergleich unternehmerischer Nachhaltigkeitsstrategien (GIVUN)“ (2015-2018) wurden Fragen der Suffizienz thematisiert, etwa bezüglich der Frage, wie stark diese bzw. eine absolute Reduktion des Ressourcenverbrauchs von verschiedenen Instrumenten unternehmerischer Verantwortung adressiert und eingefordert wird (Sommer et al. 2016). Auch das vom Umweltbundesamt geförderte Forschungsprojekt "Von der Nische in den Mainstream. Wie gute Beispiele nachhaltigen Handelns in einem breiten gesellschaftlichen Kontext verankert werden können." beschäftigte sich u.a. mit Suffizienzorientierungen und -strategien (Kny et al. 2015).
    
\subsection*{Sufizienzpolitik}
% Absatz kürzen oder raus, da allgemein
Der Begriff „Suffizienz“ ist von Wolfgang Sachs in Anlehnung an Herman Daly und Ernst Friedrich Schumacher 1993 in die Nachhaltigkeitsdebatte eingeführt und der Effizienz gegenübergestellt worden \cite{Sachs1993,Daly1991} (REFERENZ:Schumacher neue Auflage von 2013). Anders als Effizienz und Konsistenz ist Suffizienz mit einem grundlegenden Strukturwandel verbunden, der nicht ausschließlich technologische sondern vor allem gesellschaftliche Dimensionen umfasst. Oft wird er mit der moralischen Ebene (Verzicht) assoziiert und mit individuellem Konsumverhalten verknüpft und so auf eine Verantwortung des Individuums reduziert.

Als Nachhaltigkeits-Strategie mit einer politischen Dimension ist Suffizienz lange kaum debattiert worden \cite{Winterfeld2002}. Erst im Kontext der ab 2010 sich neu entzündenden wachstumskritischen Debatte wurde der Begriff vermehrt aufgegriffen und bearbeitet (REFERENZEN siehe u.a. Seidl/ Zahrnt 2010 REFERENZ; Rätz/ von Egan-Krieger 2011; und Adler/ Schachtschneider 2017), \cite{Schneidewind2013,Winterfeld2017,Linz2015,Bierwirth2015,Thomas2015}. Die internationale Debatte zu Suffizienzpolitiken wird stark durch den ökofeministischen Ansatz von REFERENZ Ariel Salleh (2009) mit geprägt und ergänzt aus der Perspektive des globalen Südens den Ansatz des „Savings“ (das Recht, etwas übrig zu behalten im Kontext der Armutsbekämpfung).

Den Stand der Technik betreffend ist immer noch festzustellen, dass sie eher suffizienzschwach ist und technischer Fortschritt stark an ökologischer Modernisierung, nicht aber an ökologischem Strukturwandel ausgerichtet ist. Auf der Akteursebene ist allerdings festzustellen, dass Suffizienz in der Architekturwissenschaft an Bedeutung gewonnen hat und hier in einschlägigen Fachzeitschriften (u.a. \cite{Architekt2015}), über Veranstaltungen bis hin zu den Architektenkammern ein Interesse an und Aufgeschlossenheit für Suffizienz auszumachen ist. 
Die Suffizienz-Forschung und Arbeit zu Suffizienzpolitik ist am Wuppertal Institut seit vielen Jahren verankert (vgl. Verweise auf Wolfgang Sachs, Manfred Linz, Uwe Schneidewind u.a.). Neben (Beiträgen zu) den genannten Publikationen und Veranstaltungen umfassten die Arbeiten zu diesem Thema in den letzten Jahren insbesondere das Projekt „Energiesuffizienz“ \cite{EnergiesuffizienzProjekt} im Rahmen der Sozial-ökologischen Forschung des BMBF sowie das Projekt „Policy Guide: Energy Sufficiency in Buildings“ \cite{EnergySufficiencyProjekt} zur Entwicklung suffizienzpolitischer Maßnahmen für die europäische Ebene. Darüber hinaus bestehen umfangreiche Erfahrungen in der Bearbeitung inter- und transdisziplinärer Projekte (z.B. im Rahmen der Entwicklung des Klimaschutzplans NRW (Suffizienz wurde als Klimaschutzsstrategie in der Arbeitsgruppe „Private Haushalte“ diskutiert, welche u.a. von Benjamin Best begleitet wurde. \url{https://wupperinst.org/p/wi/p/s/pd/376/} und \url{https://wupperinst.org/p/wi/p/s/pd/396/}, interdisziplinärer Modellierung z.B. im Rahmen des Projektes COMBI \url{https://combi-project.eu/} sowie in der methodischen Entwicklung zur Integration sozialwissenschaftlicher Empirie in die Modellierung \cite{Bierwirth2016}.\\

% weitere Veröff Projekt Energiesuffizienz
%https://energiesuffizienz.files.wordpress.com/2014/10/thema-kg-analyse.pdf
%https://energiesuffizienz.files.wordpress.com/2014/12/arbeitspapier-breitenbefragung-160513.pdf
%Zu Politiken auch:
%https://energiesuffizienz.files.wordpress.com/2014/06/energiesuffizienzpolitik_20161212.pdf

\subsection*{Überleitung zu Abschnitt 3}\\
Während die Integration von Suffizienzpolitiken und Transformationserzählungen in Ansätzen schon erfolgt, stellt die suffizienzbasierte Modellierung eine noch kaum abschätzbare Herausforderung dar \cite{Samadi2017}. Eine gelungene inter- und transdisziplinäre Verkopplung aller drei Teilbereiche bedarf neben neuer Theorieansätze (z.B. zur „Externalisierung“) auch neuer methodologischer und methodischer Ansätze einschließlich quantitativ-qualitativer Verkopplungen.