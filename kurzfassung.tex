Energiesystem-Modelle sind etablierte Werkzeuge, um technisch mögliche und ökonomisch vorteilhafte Energiewende-Pfade im Zusammenspiel von Effizienz- und Konsistenz-Strategien abzubilden. Entwicklungen, die eine Reduktion des absoluten Energieverbrauchs durch veränderte Praktiken und Routinen im Sinne einer Suffizienz-Strategie ermöglichen, werden bisher in den Modellen kaum berücksichtigt.
Ziel der inter- und transdisziplinär arbeitenden Forschungsgruppe ist es, Suffizienzaspekte und -strategien für die Energiesystem-Modellierung zu operationalisieren und damit handlungsbasierte Parameter und gesellschaftlichen Wandel in Energie- und Klimaschutzszenarien darstellbar zu machen. Die Parameter werden dabei mit Instrumenten, Handlungsoptionen und veränderten Rahmenbedingungen explizit hinterlegt. Hierfür gilt es, gesellschaftliche Transformationsprozesse im Kontext der Energiewende besser zu verstehen, die Blockaden für und Potenziale von Suffizienzpolitiken auszuloten und nicht zuletzt im Sinne einer globalen Nachhaltigkeit mögliche Externalisierungs- und Verlagerungseffekte zu diskutieren.
Die erarbeiteten Methoden zur konsistenten, integrativen und zur qualitativen Seite hin offenen Szenarienerstellung kann von anderen Forscher*innen angewendet und weiterentwickelt werden. Auch das Energiesystem-Modell, dessen Programmier-Code und Daten open source zur Verfügung gestellt wird, bildet eine Basis um auf den Erkenntnissen zu Energie-Suffizienz aufzubauen. Darüber hinaus wird durch die historische Rekonstruktion und die auf die Zukunft gerichteten Szenarien Transformationswissen generiert werden, das in konkretes Handeln umgesetzt werden kann. Gewonnene Erkenntnisse der Nachwuchsforschungsgruppe zu Blockaden für und Potenzialen von Energiesuffizienz-Politiken tragen zu politischen wie zivilgesellschaftlichen Transformations- und sozialen Innovationsprozessen bei.

