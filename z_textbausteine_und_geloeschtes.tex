\cite{Schneidewind2013} legen dar wie Suffzizienzstragien sich in die politische Praxis übersetzen lassen.
Die von \cite{Winterfeld2007} postulierte These, dass Suffizienz im Dreigestirn der Nachhaltigkeit die schwächste Position besetzt gilt noch immer, laut \cite{Winterfeld2007}, da sie mit der vorherrschenden Logik nicht kompatibel ist.
Eine sozialwissenschaftliche Dissertation wagt sich an eine empirische Fundierung von Suffizienz über in deutschen Privathaushalten über Haushalsinterviews \cite{Speck2016}.

% weitere WUppertal-Links:
% https://epub.wupperinst.org/frontdoor/index/index/docId/1512 (2002, große Übersicht)
% https://epub.wupperinst.org/frontdoor/index/index/docId/5420 (Kopatz, kurz, Energiewende, Baubeispiel.


%%%%%%%%%%%%%%%%%%%%%
Norbert Elias Center for Transformation Design & Research (NEC)
Das NEC der Europa-Universität Flensburg erforscht vor dem Hintergrund von Klimawandel, Ressourcenverknappung und Umweltverschmutzung theoriegeleitet und zugleich praxisnah die Möglichkeiten zur gesellschaftlichen Veränderung unter dem Leitbild der Zukunftsfähigkeit. Der Soziologe Bernd Sommer verantwortet am NEC den Forschungsbereich „Klima, Kultur und Nachhaltigkeit“ und leitet u.a. das BMBF-geförderte Projekt „Gemeinwohl-Ökonomie im Kontext unternehmerischer Nachhaltigkeitsstrategien“ (GIVUN). Zwischen 2009 und 2012 war Sommer am KWI tätigt, wo er u.a. das Projekt „Change Agents für den Klimaschutz“ leitete und das Graduiertenkolleg „Herausforderung der Demokratie durch den Klimawandel“ (gefördert durch die Hans-Böckler-Stiftung) koordinierte. Als Referent des WBGU war er an der Erstellung des Hauptgutachtens „Welt im Wandel. Gesellschaftsvertrag für die Große Transformation“ (2011), des Sondergutachtens „Kassensturz für den Weltklimavertrag – Der Budgetansatz“ (2009) sowie verschiedener Politikpapiere beteiligt. Sommer hat zahlreiche Veröffentlichungen zu den Themen Klimawandel, Klimaschutz, nachhaltiger Konsum und sozial-ökologische Transformation vorgelegt (siehe Publikationsliste in der Anlage). Mit dem Konzept des Transformationsdesigns (Sommer und Welzer 2014) hat er zusammen mit Harald Welzer das Forschungsprogramm eines gestalterischen Ansatzes im Feld der sozial-ökologischen Forschung beschrieben. Die Soziologin Michaela Christ leitet am NEC den Forschungsbereich „Diachrone Transformationsforschung“. In ihrem gegenwärtigen Forschungsprojekt beschäftigt sie sich mit dem Zusammenhang zwischen künstlicher Beleuchtung, wirtschaftlichem Wachstum und sozial-ökologischen Krisen. Michaela Christ ist Mitglied im deutsch-französischen Forschungsnetzwerk „Neue Evaluations- und Bewertungsrahmen zentraler gesellschaftlicher Veränderungen“ am Wissenschaftszentrum Berlin, in dem unter anderem urbane Transformationsprozesse untersucht werden. Für das geplante Forschungsvorhaben sind insbesondere die folgenden Projekte der Antragsteller*innen von Relevanz:
⎯	Zwei Grad mehr in Deutschland – Das Szenario 2040 (gefördert durch die Stiftung Forum für Verantwortung; Leitung: Harald Welzer und Friedrich-Wilhelm Gerstengarbe, Potsdam-Institut für Klimafolgenforschung, PIK; Laufzeit: Januar 2012 - Dezember 2013): Klimaforscher des PIK haben gemeinsam mit Sozialwissenschaftlern des NEC ein konkretes Wirkungsszenario der Erwärmung für Deutschland um 2040 erarbeitet.
⎯	Wissensbasis für individuelles Handeln – Change Agents für den Klimaschutz (gefördert durch den KlimaKreis Köln; Leitung: Claus Leggewie und Bernd Sommer; Laufzeit: Januar 2011 - Dezember 2012): Ziel des Forschungsprojektes war die Entwicklung adäquater Handlungsvorschläge, um an lokales Sozialkapital anzuknüpfen und zentrale Akteure für eine nachhaltige Stadtteilentwicklung zu gewinnen. 
⎯	Von der Nische in den Mainstream. Wie gute Beispiele nachhaltigen Handelns in einem breiten gesellschaftlichen Kontext verankert werden können (gefördert durch das Umweltbundesamt; Leitung: Harald Welzer und Bernd Sommer; Laufzeit: Juni 2013 - Juli 2014): Im Rahmen des Forschungsvorhabens sind Indikatoren für nachhaltiges Handeln und Kriterien für eine erfolgreiche Diffusion erarbeitet worden. 
Für weitere Informationen zum NEC, den Forschungsprojekten sowie zu den Antragssteller*innen siehe: www.norberteliascenter.de.

%%%%%%%%%%%%%%%%%%%%%5

\subsection*{Zukunftsperspektiven}
\begin{itemize}
 \item Leitung: Professur in einem Bereich der bisher nicht ausreichend besetzt ist, in einem immer bedeutsam werdenden Forschungsfeld, evtl. direkt am interdisziplin. Institut umsetzbar?
 \item Doktoranden: Voraussetzung für weitere sozial-ökologische interdisziplinäre Forschungslaufbahn geschaffen. Auslandsaufenthalt, Methodenkomptenz, Lehrerfahrung hilft dabei.
 \item weitere beteiligte PostDocs: Interdisziplinär fortgebildet, erweitert den Blick für bisher nicht besetzte Forschungsfelder, neue Projekte, neue Netzwerke, neue Forschungsideen
\end{itemize}

\subsection*{Anwendungspotential, Ergebnisverwertung, Öffentlichkeitsarbeit, Wissenstransfer}
\begin{itemize}
 \item Teilnahme an Formaten wie Science Slam
 \item evtl. ein Blog?
 \item wissenschaftliche Publikationen, erst disziplinär, dann auch interdisziplinär (oft schwieriger)
 \item Lernziel ist auch nicht-wissenschaftlich schreiben zu lernen, die Forschung kommunizieren zu können
 \item Ringvorlesung Präsentationen
 \item evtl. Exkursionsreihe 'Suffizienz-Beispiele in der Praxis' in Flensburg?
 \item Konferenzen
 \item evtl. die Output-Punkte von oben übernehmen
 \item Modell liegt vor, wird auf bekannten code-sharing platforms veröffentlicht, weiterentwicklung ist transparent, auch andere können vorschläge machen, kritisieren, beitragen
\end{itemize}

%%%%%%%%%%%%%%%%%%%%%%%%%%%%%%%%

% Jonas: Sprich wir (D/der Globale Norden) müssen unseren Energieverbrauch nicht nur aufgrund von einer Klimakrise senken, sondern auch z.B. aus Gründen des Ressourcenbedarfs für EE (z.B. Flächen in D - dazu findet sich auch etwas in einer Studie vom Fraunhofer, die ich dir angehängt hatte, seltene Erden usw.). Darüber hinaus basieren die gegenwärtigen Rückgänge des Energieverbrauchs nicht allein auf Effizienzerfolgen, sondern auch auf Verlagerungseffekten, sind also nur eine scheinbare Reduktion (vgl. auch mein Kommentar und die Quelle zum material foodprint). Hier könnte auch mit Rebound-Effekten argumentiert werden, dass Effizienz alleine nicht ausreicht. 

% Olav:In Anbetracht der möglichen Energiebereitstellung durch regenerative Energiequellen, sind wir meiner Überzeugung nach thermodynamisch noch weit von dem Punkt entfernt, wo das unausweichlich ist. Es geht wohl vielmehr um das Kostenminimum oder Wohlfahrtsmaximum im Zielraum der möglichen nachhaltigen Entwicklungspfade, für die keine der drei Dimensionen außer Acht gelassen werden darf.

% Bernd: Hier ggf. darauf verweisen, dass Modellrechnungen/Szenarien dies nahe legen. Grundsätzlich erscheint es mir wichtig, möglichst gut zu begründen zu machen, dass die ambitionierten Energie- und Klimaziele (nach dem Pariser Abkommen) nicht ohne Energie-Suffizienz zu erreichen sein werden. Denn das Thema "Suffizienz" (so wissen wir aus eigener Erfahrung) hat im BMBF nicht nur Freunde. D.h. es muss dargelegt werden, dass Effizienz- und Konsistenzstrategien alleine nicht reichen werden. Unten (zur Rebound-Thematik) habe ich einen Textbaustein aus einem alten Antrag einkopiert, der hierzu passt und den Du an geeigneter Stelle gerne verwenden kannst.

Die Herausforderung, die ambitionierten Klimaziele des Pariser Klimaabkommens zu erreichen (doi:10.1038/s41558-018-0091-3) und die planetaren Grenzen langfristig einzuhalten (doi:10.1038/461472a), wird jedoch nur im Zusammenspiel von Konsistenz (Erneuerbare Energien ersetzen Fossile), Effizienz (relative Reduktion des Energieverbrauchs bei Bereitstellung der gleichen Energiedienstleistung) und Suffizienz (absolute Reduktion der Nachfrage nach Energiedienstleistungen durch veränderte soziale Praktiken ohne Einbuße des menschlichen Wohlbefindens) gelingen. Daher ist es notwendig alle drei Nachhaltigkeitsstrategien auch im Zusammenhang zu betrachten. Die Rolle von Energie-Suffizienz ist derzeit jedoch unterrepräsentiert in Diskussion und Forschung für Klimaschutz und Energiewende. Abbildung \ref{fig:zusammenspiel} verdeutlicht den Gedanken, dass -auch wenn die bisherige Emissionsreduktion in Deutschland durch Konsistenz-Maßnahmen gelungen ist- in Zukunft der Beitrag von Effizienz und Suffizienz unverzichtbar für die Erreichung der Klimaziel ist.

%%%%%%%%%%%%%%%%%%%

zum NEC:
HIer ein Baustein zum NEC, der aber um unser Suffizienz-Projekt (EHSS) ergänzt werden muss:

Norbert Elias Center for Transformation Design & Research (NEC)
Das NEC der Europa-Universität Flensburg erforscht vor dem Hintergrund von Klimawandel, Ressourcenverknappung und Umweltverschmutzung theoriegeleitet und zugleich praxisnah die Möglichkeiten zur gesellschaftlichen Veränderung unter dem Leitbild der Zukunftsfähigkeit. Der Soziologe Bernd Sommer verantwortet am NEC den Forschungsbereich „Klima, Kultur und Nachhaltigkeit“ und leitet u.a. das BMBF-geförderte Projekt „Gemeinwohl-Ökonomie im Kontext unternehmerischer Nachhaltigkeitsstrategien“ (GIVUN). Zwischen 2009 und 2012 war Sommer am KWI tätigt, wo er u.a. das Projekt „Change Agents für den Klimaschutz“ leitete und das Graduiertenkolleg „Herausforderung der Demokratie durch den Klimawandel“ (gefördert durch die Hans-Böckler-Stiftung) koordinierte. Als Referent des WBGU war er an der Erstellung des Hauptgutachtens „Welt im Wandel. Gesellschaftsvertrag für die Große Transformation“ (2011), des Sondergutachtens „Kassensturz für den Weltklimavertrag – Der Budgetansatz“ (2009) sowie verschiedener Politikpapiere beteiligt. Sommer hat zahlreiche Veröffentlichungen zu den Themen Klimawandel, Klimaschutz, nachhaltiger Konsum und sozial-ökologische Transformation vorgelegt (siehe Publikationsliste in der Anlage). Mit dem Konzept des Transformationsdesigns (Sommer und Welzer 2014) hat er zusammen mit Harald Welzer das Forschungsprogramm eines gestalterischen Ansatzes im Feld der sozial-ökologischen Forschung beschrieben. Die Soziologin Michaela Christ leitet am NEC den Forschungsbereich „Diachrone Transformationsforschung“. In ihrem gegenwärtigen Forschungsprojekt beschäftigt sie sich mit dem Zusammenhang zwischen künstlicher Beleuchtung, wirtschaftlichem Wachstum und sozial-ökologischen Krisen. Michaela Christ ist Mitglied im deutsch-französischen Forschungsnetzwerk „Neue Evaluations- und Bewertungsrahmen zentraler gesellschaftlicher Veränderungen“ am Wissenschaftszentrum Berlin, in dem unter anderem urbane Transformationsprozesse untersucht werden. Für das geplante Forschungsvorhaben sind insbesondere die folgenden Projekte der Antragsteller*innen von Relevanz:
⎯	Zwei Grad mehr in Deutschland – Das Szenario 2040 (gefördert durch die Stiftung Forum für Verantwortung; Leitung: Harald Welzer und Friedrich-Wilhelm Gerstengarbe, Potsdam-Institut für Klimafolgenforschung, PIK; Laufzeit: Januar 2012 - Dezember 2013): Klimaforscher des PIK haben gemeinsam mit Sozialwissenschaftlern des NEC ein konkretes Wirkungsszenario der Erwärmung für Deutschland um 2040 erarbeitet.
⎯	Wissensbasis für individuelles Handeln – Change Agents für den Klimaschutz (gefördert durch den KlimaKreis Köln; Leitung: Claus Leggewie und Bernd Sommer; Laufzeit: Januar 2011 - Dezember 2012): Ziel des Forschungsprojektes war die Entwicklung adäquater Handlungsvorschläge, um an lokales Sozialkapital anzuknüpfen und zentrale Akteure für eine nachhaltige Stadtteilentwicklung zu gewinnen. 
⎯	Von der Nische in den Mainstream. Wie gute Beispiele nachhaltigen Handelns in einem breiten gesellschaftlichen Kontext verankert werden können (gefördert durch das Umweltbundesamt; Leitung: Harald Welzer und Bernd Sommer; Laufzeit: Juni 2013 - Juli 2014): Im Rahmen des Forschungsvorhabens sind Indikatoren für nachhaltiges Handeln und Kriterien für eine erfolgreiche Diffusion erarbeitet worden. 
Für weitere Informationen zum NEC, den Forschungsprojekten sowie zu den Antragssteller*innen siehe: www.norberteliascenter.de.
