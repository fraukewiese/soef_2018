\cite{Schneidewind2013} legen dar wie Suffzizienzstragien sich in die politische Praxis übersetzen lassen.
Die von \cite{Winterfeld2007} postulierte These, dass Suffizienz im Dreigestirn der Nachhaltigkeit die schwächste Position besetzt gilt noch immer, laut \cite{Winterfeld2007}, da sie mit der vorherrschenden Logik nicht kompatibel ist.
Eine sozialwissenschaftliche Dissertation wagt sich an eine empirische Fundierung von Suffizienz über in deutschen Privathaushalten über Haushalsinterviews \cite{Speck2016}.

% weitere WUppertal-Links:
% https://epub.wupperinst.org/frontdoor/index/index/docId/1512 (2002, große Übersicht)
% https://epub.wupperinst.org/frontdoor/index/index/docId/5420 (Kopatz, kurz, Energiewende, Baubeispiel.


%%%%%%%%%%%%%%%%%%%%%
Norbert Elias Center for Transformation Design & Research (NEC)
Das NEC der Europa-Universität Flensburg erforscht vor dem Hintergrund von Klimawandel, Ressourcenverknappung und Umweltverschmutzung theoriegeleitet und zugleich praxisnah die Möglichkeiten zur gesellschaftlichen Veränderung unter dem Leitbild der Zukunftsfähigkeit. Der Soziologe Bernd Sommer verantwortet am NEC den Forschungsbereich „Klima, Kultur und Nachhaltigkeit“ und leitet u.a. das BMBF-geförderte Projekt „Gemeinwohl-Ökonomie im Kontext unternehmerischer Nachhaltigkeitsstrategien“ (GIVUN). Zwischen 2009 und 2012 war Sommer am KWI tätigt, wo er u.a. das Projekt „Change Agents für den Klimaschutz“ leitete und das Graduiertenkolleg „Herausforderung der Demokratie durch den Klimawandel“ (gefördert durch die Hans-Böckler-Stiftung) koordinierte. Als Referent des WBGU war er an der Erstellung des Hauptgutachtens „Welt im Wandel. Gesellschaftsvertrag für die Große Transformation“ (2011), des Sondergutachtens „Kassensturz für den Weltklimavertrag – Der Budgetansatz“ (2009) sowie verschiedener Politikpapiere beteiligt. Sommer hat zahlreiche Veröffentlichungen zu den Themen Klimawandel, Klimaschutz, nachhaltiger Konsum und sozial-ökologische Transformation vorgelegt (siehe Publikationsliste in der Anlage). Mit dem Konzept des Transformationsdesigns (Sommer und Welzer 2014) hat er zusammen mit Harald Welzer das Forschungsprogramm eines gestalterischen Ansatzes im Feld der sozial-ökologischen Forschung beschrieben. Die Soziologin Michaela Christ leitet am NEC den Forschungsbereich „Diachrone Transformationsforschung“. In ihrem gegenwärtigen Forschungsprojekt beschäftigt sie sich mit dem Zusammenhang zwischen künstlicher Beleuchtung, wirtschaftlichem Wachstum und sozial-ökologischen Krisen. Michaela Christ ist Mitglied im deutsch-französischen Forschungsnetzwerk „Neue Evaluations- und Bewertungsrahmen zentraler gesellschaftlicher Veränderungen“ am Wissenschaftszentrum Berlin, in dem unter anderem urbane Transformationsprozesse untersucht werden. Für das geplante Forschungsvorhaben sind insbesondere die folgenden Projekte der Antragsteller*innen von Relevanz:
⎯	Zwei Grad mehr in Deutschland – Das Szenario 2040 (gefördert durch die Stiftung Forum für Verantwortung; Leitung: Harald Welzer und Friedrich-Wilhelm Gerstengarbe, Potsdam-Institut für Klimafolgenforschung, PIK; Laufzeit: Januar 2012 - Dezember 2013): Klimaforscher des PIK haben gemeinsam mit Sozialwissenschaftlern des NEC ein konkretes Wirkungsszenario der Erwärmung für Deutschland um 2040 erarbeitet.
⎯	Wissensbasis für individuelles Handeln – Change Agents für den Klimaschutz (gefördert durch den KlimaKreis Köln; Leitung: Claus Leggewie und Bernd Sommer; Laufzeit: Januar 2011 - Dezember 2012): Ziel des Forschungsprojektes war die Entwicklung adäquater Handlungsvorschläge, um an lokales Sozialkapital anzuknüpfen und zentrale Akteure für eine nachhaltige Stadtteilentwicklung zu gewinnen. 
⎯	Von der Nische in den Mainstream. Wie gute Beispiele nachhaltigen Handelns in einem breiten gesellschaftlichen Kontext verankert werden können (gefördert durch das Umweltbundesamt; Leitung: Harald Welzer und Bernd Sommer; Laufzeit: Juni 2013 - Juli 2014): Im Rahmen des Forschungsvorhabens sind Indikatoren für nachhaltiges Handeln und Kriterien für eine erfolgreiche Diffusion erarbeitet worden. 
Für weitere Informationen zum NEC, den Forschungsprojekten sowie zu den Antragssteller*innen siehe: www.norberteliascenter.de.

%%%%%%%%%%%%%%%%%%%%%5

\subsection*{Zukunftsperspektiven}
\begin{itemize}
 \item Leitung: Professur in einem Bereich der bisher nicht ausreichend besetzt ist, in einem immer bedeutsam werdenden Forschungsfeld, evtl. direkt am interdisziplin. Institut umsetzbar?
 \item Doktoranden: Voraussetzung für weitere sozial-ökologische interdisziplinäre Forschungslaufbahn geschaffen. Auslandsaufenthalt, Methodenkomptenz, Lehrerfahrung hilft dabei.
 \item weitere beteiligte PostDocs: Interdisziplinär fortgebildet, erweitert den Blick für bisher nicht besetzte Forschungsfelder, neue Projekte, neue Netzwerke, neue Forschungsideen
\end{itemize}

\subsection*{Anwendungspotential, Ergebnisverwertung, Öffentlichkeitsarbeit, Wissenstransfer}
\begin{itemize}
 \item Teilnahme an Formaten wie Science Slam
 \item evtl. ein Blog?
 \item wissenschaftliche Publikationen, erst disziplinär, dann auch interdisziplinär (oft schwieriger)
 \item Lernziel ist auch nicht-wissenschaftlich schreiben zu lernen, die Forschung kommunizieren zu können
 \item Ringvorlesung Präsentationen
 \item evtl. Exkursionsreihe 'Suffizienz-Beispiele in der Praxis' in Flensburg?
 \item Konferenzen
 \item evtl. die Output-Punkte von oben übernehmen
 \item Modell liegt vor, wird auf bekannten code-sharing platforms veröffentlicht, weiterentwicklung ist transparent, auch andere können vorschläge machen, kritisieren, beitragen
\end{itemize}

%%%%%%%%%%%%%%%%%%%%%%%%%%%%%%%%

% Jonas: Sprich wir (D/der Globale Norden) müssen unseren Energieverbrauch nicht nur aufgrund von einer Klimakrise senken, sondern auch z.B. aus Gründen des Ressourcenbedarfs für EE (z.B. Flächen in D - dazu findet sich auch etwas in einer Studie vom Fraunhofer, die ich dir angehängt hatte, seltene Erden usw.). Darüber hinaus basieren die gegenwärtigen Rückgänge des Energieverbrauchs nicht allein auf Effizienzerfolgen, sondern auch auf Verlagerungseffekten, sind also nur eine scheinbare Reduktion (vgl. auch mein Kommentar und die Quelle zum material foodprint). Hier könnte auch mit Rebound-Effekten argumentiert werden, dass Effizienz alleine nicht ausreicht. 

% Olav:In Anbetracht der möglichen Energiebereitstellung durch regenerative Energiequellen, sind wir meiner Überzeugung nach thermodynamisch noch weit von dem Punkt entfernt, wo das unausweichlich ist. Es geht wohl vielmehr um das Kostenminimum oder Wohlfahrtsmaximum im Zielraum der möglichen nachhaltigen Entwicklungspfade, für die keine der drei Dimensionen außer Acht gelassen werden darf.

% Bernd: Hier ggf. darauf verweisen, dass Modellrechnungen/Szenarien dies nahe legen. Grundsätzlich erscheint es mir wichtig, möglichst gut zu begründen zu machen, dass die ambitionierten Energie- und Klimaziele (nach dem Pariser Abkommen) nicht ohne Energie-Suffizienz zu erreichen sein werden. Denn das Thema "Suffizienz" (so wissen wir aus eigener Erfahrung) hat im BMBF nicht nur Freunde. D.h. es muss dargelegt werden, dass Effizienz- und Konsistenzstrategien alleine nicht reichen werden. Unten (zur Rebound-Thematik) habe ich einen Textbaustein aus einem alten Antrag einkopiert, der hierzu passt und den Du an geeigneter Stelle gerne verwenden kannst.

Die Herausforderung, die ambitionierten Klimaziele des Pariser Klimaabkommens zu erreichen (doi:10.1038/s41558-018-0091-3) und die planetaren Grenzen langfristig einzuhalten (doi:10.1038/461472a), wird jedoch nur im Zusammenspiel von Konsistenz (Erneuerbare Energien ersetzen Fossile), Effizienz (relative Reduktion des Energieverbrauchs bei Bereitstellung der gleichen Energiedienstleistung) und Suffizienz (absolute Reduktion der Nachfrage nach Energiedienstleistungen durch veränderte soziale Praktiken ohne Einbuße des menschlichen Wohlbefindens) gelingen. Daher ist es notwendig alle drei Nachhaltigkeitsstrategien auch im Zusammenhang zu betrachten. Die Rolle von Energie-Suffizienz ist derzeit jedoch unterrepräsentiert in Diskussion und Forschung für Klimaschutz und Energiewende. Abbildung \ref{fig:zusammenspiel} verdeutlicht den Gedanken, dass -auch wenn die bisherige Emissionsreduktion in Deutschland durch Konsistenz-Maßnahmen gelungen ist- in Zukunft der Beitrag von Effizienz und Suffizienz unverzichtbar für die Erreichung der Klimaziel ist.

%%%%%%%%%%%%%%%%%%%

zum NEC:
HIer ein Baustein zum NEC, der aber um unser Suffizienz-Projekt (EHSS) ergänzt werden muss:

Norbert Elias Center for Transformation Design & Research (NEC)
Das NEC der Europa-Universität Flensburg erforscht vor dem Hintergrund von Klimawandel, Ressourcenverknappung und Umweltverschmutzung theoriegeleitet und zugleich praxisnah die Möglichkeiten zur gesellschaftlichen Veränderung unter dem Leitbild der Zukunftsfähigkeit. Der Soziologe Bernd Sommer verantwortet am NEC den Forschungsbereich „Klima, Kultur und Nachhaltigkeit“ und leitet u.a. das BMBF-geförderte Projekt „Gemeinwohl-Ökonomie im Kontext unternehmerischer Nachhaltigkeitsstrategien“ (GIVUN). Zwischen 2009 und 2012 war Sommer am KWI tätigt, wo er u.a. das Projekt „Change Agents für den Klimaschutz“ leitete und das Graduiertenkolleg „Herausforderung der Demokratie durch den Klimawandel“ (gefördert durch die Hans-Böckler-Stiftung) koordinierte. Als Referent des WBGU war er an der Erstellung des Hauptgutachtens „Welt im Wandel. Gesellschaftsvertrag für die Große Transformation“ (2011), des Sondergutachtens „Kassensturz für den Weltklimavertrag – Der Budgetansatz“ (2009) sowie verschiedener Politikpapiere beteiligt. Sommer hat zahlreiche Veröffentlichungen zu den Themen Klimawandel, Klimaschutz, nachhaltiger Konsum und sozial-ökologische Transformation vorgelegt (siehe Publikationsliste in der Anlage). Mit dem Konzept des Transformationsdesigns (Sommer und Welzer 2014) hat er zusammen mit Harald Welzer das Forschungsprogramm eines gestalterischen Ansatzes im Feld der sozial-ökologischen Forschung beschrieben. Die Soziologin Michaela Christ leitet am NEC den Forschungsbereich „Diachrone Transformationsforschung“. In ihrem gegenwärtigen Forschungsprojekt beschäftigt sie sich mit dem Zusammenhang zwischen künstlicher Beleuchtung, wirtschaftlichem Wachstum und sozial-ökologischen Krisen. Michaela Christ ist Mitglied im deutsch-französischen Forschungsnetzwerk „Neue Evaluations- und Bewertungsrahmen zentraler gesellschaftlicher Veränderungen“ am Wissenschaftszentrum Berlin, in dem unter anderem urbane Transformationsprozesse untersucht werden. Für das geplante Forschungsvorhaben sind insbesondere die folgenden Projekte der Antragsteller*innen von Relevanz:
⎯	Zwei Grad mehr in Deutschland – Das Szenario 2040 (gefördert durch die Stiftung Forum für Verantwortung; Leitung: Harald Welzer und Friedrich-Wilhelm Gerstengarbe, Potsdam-Institut für Klimafolgenforschung, PIK; Laufzeit: Januar 2012 - Dezember 2013): Klimaforscher des PIK haben gemeinsam mit Sozialwissenschaftlern des NEC ein konkretes Wirkungsszenario der Erwärmung für Deutschland um 2040 erarbeitet.
⎯	Wissensbasis für individuelles Handeln – Change Agents für den Klimaschutz (gefördert durch den KlimaKreis Köln; Leitung: Claus Leggewie und Bernd Sommer; Laufzeit: Januar 2011 - Dezember 2012): Ziel des Forschungsprojektes war die Entwicklung adäquater Handlungsvorschläge, um an lokales Sozialkapital anzuknüpfen und zentrale Akteure für eine nachhaltige Stadtteilentwicklung zu gewinnen. 
⎯	Von der Nische in den Mainstream. Wie gute Beispiele nachhaltigen Handelns in einem breiten gesellschaftlichen Kontext verankert werden können (gefördert durch das Umweltbundesamt; Leitung: Harald Welzer und Bernd Sommer; Laufzeit: Juni 2013 - Juli 2014): Im Rahmen des Forschungsvorhabens sind Indikatoren für nachhaltiges Handeln und Kriterien für eine erfolgreiche Diffusion erarbeitet worden. 
Für weitere Informationen zum NEC, den Forschungsprojekten sowie zu den Antragssteller*innen siehe: www.norberteliascenter.de.

%%%%%%%%%%%
REBOUND
-        Druckman, Angela/ Chitnis, Mona/Sorrell, Steve/Jackson, Tim (2011): Missing  carbon reductions? Exploring rebound and backfire effects in UK households. In: Energy Policy 39: 3572-3581.
 

-        Sorrell, Steve (2007): The rebound effect: an assessment of the evidence for economy‐wide energy savings from improved energy efficiency. London: UKERC.

-        Sorrell, Setve/Dimitropoulos, John/ Sommerville, Matt (2009): Empirical estimates of the direct rebound effect: A review. In: Energy Policy 37(4): 1356‐1371.

In der Fachliteratur werden grob zwei Varianten von Rebound-Effekten unterschieden (Sorrell 2007): Zum einen ist zu beobachten, dass technische Geräte – beispielsweise Kühlschränke oder Fahrzeuge – zwar effizienter, aber zugleich vielfach auch größer werden und/oder häufiger genutzt werden und infolgedessen genau so viel Energie verbrauchen wie ihre weniger effizienten Vorgängermodelle oder sogar mehr (direkter Rebound-Effekt). Zum anderen wird das durch effizientere Technologien eingesparte Geld von Verbrauchern vielfach für andere, z. T. noch energieintensivere Aktivitäten verwendet. Dies ist beispielsweise der Fall, wenn die Rückzahlung der Heizkosten, die das Resultat einer energetischen Haussanierung ist, zur Anschaffung eines Zweitfernsehers genutzt wird (indirekter Rebound-Effekt). Während bereits eine Reihe von Studien belastbare Daten zur quantitativen Abschätzung direkter Rebound-Effekte bereitstellen (siehe z. B. Sorrell 2009 et al.), gestaltet sich die Ermittlung indirekter Rebound-Effekte in der Forschung sehr viel schwieriger, und die Anzahl entsprechender Studien ist vergleichsweise gering (Druckman et al. 2011).

%%%%%%%%
KURZFASSUNG:

\section*{Kurzfassung}
Die Herausforderung, Klimaziele zu erreichen und die Grenzen des Planeten langfristig einzuhalten, wird nur im Zusammenspiel von Konsistenz (Erneuerbare Energien ersetzen Fossile), Effizienz (relative Reduktion des Energieverbrauchs bei Bereitstellung der gleichen Energie-Dienstleistung) und Suffizienz (absolute Reduktion der Nachfrage nach Energiedienstleistungen ohne Einbuße des menschlichen Wohlbefinden) gelingen. Die Rolle von Energie-Suffizienz ist derzeit jedoch unterrepräsentiert in Diskussion, Forschung und Politik für Klimaschutz und Energiewende. Obwohl die zukünftige Nachfrage nach Strom, Wärme und Mobilität ein entscheidender Einflussparameter in Energie-System-Modellen ist, wird die Nachfrage-Seite meist nur innerhalb enger Grenzen variiert und der Fokus auf die Bereitstellung der Energie-Dienstleistungen gelegt.
In der geplanten Nachwuchs-Forschungsgruppe "Die Rolle von Energie-Suffizienz in Gesellschaft und Energiewende" wird der Zusammenhang zwischen gesellschaftlichem Wandel und Nachfrage nach Strom, Wärme und Mobilität analysiert. Darauf aufbauend werden Szenarien gesellschaftlicher Zukünfte mit Energie-Szenarien und deren quantitativer Modellierung verknüpft. Ziel der interdisziplinären Forschungsgruppe ist (1) die Entwicklung konsistenter Energie-Szenarien, die gesellschaftliche Entwicklungen über technologischen Fortschritt hinaus einbeziehen und (2) ein open source Energiesystem-Modell, dass Konsistenz-, Effizienz- und Suffzienz-Maßnahmen abbilden kann. Der gesellschaftliche Beitrag der Zusammenarbeit von sozialwissenschaftlicher Transformationsforschung und technisch-ökonomischer Energiesystem-Analyse sollen letztendlich Handlungsempfehlungen für Kommunal- und Bundespolitik bezüglich Energie-Suffizienz sein.

%Im Projekt wird der Einfluss von gesellschaftlichen Veränderungen auf die Nachfrage an Energie-Dienstleistungen untersucht. Dabei werden historische Trends analysiert und Zukünfte skizziert. Es werden Parameter identifiziert, die gesellschaftlichen Wandel in Energie-Nachfrage übersetzen können. Diese Nachfrage ist wiederum der Input für ein Energiesystem-Modell, mit dem dann der Einfluss von gesellschaftlicher Transformation auf die Ausgestaltung des Energiesystems analysiert werden kann.
%Entwickelt wird ein Rahmen für die integrierte Betrachtung von Transformation der Gesellschaft und des Energiesystems. Dabei ist die Nachfrage nach Energie-Systemdienstleistungen der verbindende Parameter. Im Anschluss an die Analyse der Rolle von Energiesuffizienz zur Erreichung von Klimazielen, werden Energiewende- und Klimaschutz-Maßnahmen und Gesetzgebung auf ihre Wirksamkeit in Bezug auf Energie-Suffizienz untersucht.
%Da die skizzierte Forschung auf Erkenntnissen aus Energiesystem-Analyse, Zukunftsforschung und Transformationsdesign aufbaut, wird sich das Team der Nachwuchsforschungsgruppe auch entsprechend interdisziplinär zusammensetzen.

%%%%%%%%%%%%%%%%%%%%%%%

PERSPEKTIVEN ZUKUNFT DER MITGLIEDER

\iffalse
Frauke Wiese möchte ihrer langjährigen Erfahrung im Bereich der Nachhaltigkeitsforschung nun
auch den entsprechenden Rahmen geben. Während ihrer Forschungsarbeit in der Energiesystem--
Modellierung reifte die Erkenntnis, dass ihre erlernte technisch-ökonomische Sichtweise nicht aus-
reicht um Fragen der Erneuerbaren Energien Systeme und des Klimaschutzes aus dem Blickwin-
kel der Nachhaltigkeit zu betrachten. Mit ihrer Dissertation zu Open Data und Open Source in der
Energiesystem-Modellierung zeigt sie die Bedeutung der Zusammenarbeit verschiedener Disziplinen
für die Erstellung von 100% Erneuerbaren Energie Szenarien auf. Die Leitung einer Nachwuchsgrup-
pe im Bereich sozial-ökologischer Forschung ermöglicht ihr den eingeschlagenen interdisziplinären
Weg weiter zu beschreiten und diesen an der Europa-Universität Flensburg aktiv mitzugestalten. So
kann sie neue Methoden sozial-ökologischer Ansätze zu vertiefen anstatt sich auf den klassischen
technisch-ökonomischen Blickwinkel zu beschränken.
\fi
%%%%%%%%%%%%%%%%%%%%%%%%%%%%%

BEZUG ZU SOZIAL-ÖKOLOGISCHER FORSCHUNG


\begin{itemize}
 \item Die Nachwuchchsforschungsgruppe stellt System-, Orientierungs- und Entscheidungswissen zu den zentralen Nachhaltigkeitsherausforderungen Energiewende und Klimaschutz bereit
 \item analysiert den Transformatinsbedarf bzgl. Energie-Suffizienz in Wirtschaft und Gesellschaft
 \item Lösungsvorschlag zu der ökologischen Krise des Klimaschutzes durch Erweiterung der Energiewende-Optionen um Suffizienz (zumindest ein Stück weiter heranrutschen an die Praxistauglichkeit damit)
 \item Wie in der sozial-ökologischen Forschung üblich, werden Analysen durchgeführt um Wechselwirkungen zwischen Gesellschaft, Wirtschaft und Umwelt abzuschätzen. Dabei wird ein Fokus auf soziale (gesellschaftliche) Entwicklungen gelegt und politische und ökonomische Entwicklungen berücksichtigt.
 \item Gemäß des Forschungsansatzes sozial-ökologischer Forschung wirken unterschiedliche Disziplinen zusammen und werden ergänzt durch Wissen aus der Praxis.
 \item will dazu beitragen realistische Lösungsoptionen für das WIE des Übergangs zu einer nachhaltigen Gesellschaft zu finden
 \item gezielte Beteiligung gesellschaftlicher Akteure, Teilhabe am Verstehen und Gestalten von Transformationsprozessen
 \item Suffizien-Handlungsempfehlungen bieten Grundlage für Veränderungsprozesse hin zu einer nachhaltigen Gesellschaft
 \item Klimaschutz und Energiewende sind zentrale Nachhaltigkeitstransformationen
 \item bei qualitativ wachsendem Wohlstand den absoluten Ressourcenverbrauch zu reduzieren, das schreit doch nach Suffizienz
 \item Politik als Adressat des in der SOEF erarbeiteten Transformationswissen und wissenschaftl. Handlungsempfehlungen
 \item in Kooperation aus Hochschule, außer und Praxis werden wissenschaftliche Grundlagen weiter entwickelt
 \item Szenarien als Methode der Integration des Wissens von verschied. Disziplinen
 \item Diskussionsrunden führender Personen aus der SOEF-Fachebene verfolgen und beiwohnen
 \item THEMATISCHER SCHWERPUNKT: co-transformations sozial-ökologischer Versorgungssysteme
 \item 3 Nachhaltig Wirtschaften hat auch was mit Suffizienz zu tun (Rebound-Effekte)
 \item SOEF im Prozess thematischer Schwerpunkt, in die Richtung der Fördermaßnahme 'Umwelt- und gesellschaftsverträgliche Transformation des Energiesystems`'
\end{itemize}

%%%%%%%%%%%%%%%%%%%

STAND WI und TE
\begin{itemize}
    \item Energiesystem-Analyse und Modellierung: wichtige Rolle auch in policy advice, hochkomplexe Modelle, inzwischen auch Open Source Modelle, wenn auch noch Mangel an Transparenz
    \item Generelle Herausforderungen ESM:
    \begin{itemize}
     \item Open heißt nicht unbedingt verständlich, bisher Mangel an Ergebniskommunikation 
     \item Sozialwissenschaftlicher Komponenten bisher unterrepräsentiert wenn es auch auf dem Gebiet Akzeptanz
     \item Arbeiten im Bereich Energie-Nachfrage beziehen sich viel auf Flexibilität der Nachfrage, weniger auf die absolute Nachfrage nach Strom, Wärme, Mobilität
    \end{itemize}
    \item Eigene Vorarbeiten (Frauke): ESM, open ESM, kritische Betrachtung ESM, Verbindung über techn.-ökono. Sichtweise hinaus, bereits interdisziplinär gearbeitet / (EUM-Team): oemof, open, Klimaschutz (transdisziplinär)
    
\end{itemize}
