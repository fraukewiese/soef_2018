\section{Kooperationen, Forschungs- und Praxispartner}
\label{sec:5}
%\textit{vorgesehene Kooperationen (Forschungs- und Praxispartner) und Arbeitsteilung, Einbindung der Praxispartner in den transdisziplinären Forschungsansatz}\\

%Weil einfach die sozial-ökologische Transformationsforschung nicht in den disziplinären Ausgangsrahmen passt, sondern es eher so ist, dass wir uns bei ihr treffen. Deshalb habe ich es mal so angeordnet. 
\begin{figure}[!h]
    \centering
    \includegraphics[width=0.9\textwidth]{figures/Konsortium4.png}
    \caption{Konsortium}
    \label{fig:konsortium}
\end{figure}

% Allgemein EUF EUM NEC Wuppertal
Die Nachwuchsgruppe ist am interdisziplinären Institut für Umwelt-, Sozial- und Humanwissenschaften der Europa-Universität Flensburg (EUF) und dem Wuppertal Institut für Klima, Umwelt, Energie angesiedelt (WI).
% Leitung geteilt
Zur Verstärkung des interdisziplinären Charakters und der engeren Kooperation zwischen Universität und außeruniversitärer Forschungseinrichtung, wird die Forschungsgruppe gemeinsam von Frauke Wiese (EUF) und Ben Best (WI) geleitet.

Der Forschungs-Schwerpunkt Energiesystem-Analyse ist in der Abteilung Energie- und Umweltmanagement (EUM) verortet, während am WI wird der Forschungs-Schwerpunkt Suffizienz-Politiken bearbeitet wird. Das NEC schlägt mit der sozial-ökologischen Transformationsforschung den inhaltlichen Bogen zwischen Wirtschaftsingenieurwesen und Politikwissenschaft. Abbildung \ref{fig:konsortium} stellt die Zusammenarbeit des Konsortiums dar.

%Die interdisziplinären Arbeitspakete werden in enger Kooperation zwischen der Universität und der außeruniversitären Forschungseinrichtung durchgeführt. In mindestens halbjährlich stattfindenden Treffen aller Mitarbeiter*innen der Forschungsgruppe werden Fortschritte diskutiert und Arbeitsschritte geplant, unterstützt von der Mentorin. Diese Gesamttreffen werden flankiert von weiteren bilateralen Arbeitstreffen und intensiven Phasen der inter und transdisziplinären Forschungsarbeit durch mehr-monatige Forschungsaufenthalte in der Partner-Institution durch die Doktoranden. Der genauere Rahmen für die Zusammenarbeit wird im ersten Jahr der Konstituierung der Gruppe gemeinsam festgelegt.

% allgemein Netzwerke Suffizienz
Weitere Kooperationen außerhalb der Verbundpartner bestehen bereits im Suffizienz-Forschungsnetzwerk, ein informeller Zusammenschluss von Forschenden im deutschsprachigen Raum, um den interdisziplinären Austausch zum Thema Suffizienz voranzutreiben und Impulse zur weiteren Untersuchung des Themas zu liefern. Weiter ist seitens des WI vorgesehen, das inter- und transdisziplinäre Netzwerk Vorsorgendes Wirtschaften einzubeziehen.

% EUM Kooperationen
Der Bereich Energiesystemanalyse bringt die bereits bestehende Forschungskooperationen mit der Open Energy Modelling Initiative \cite{openmod} ein. Da hierin so gut wie alle in Deutschland ansässigen Institutionen, die offene Energiesystem-Modellen betreiben, vertreten sind, ist dies besonders hilfreich für die Wahl und Nutzung eines der bestehenden Energiesystem-Modelle. 

% Praxispartner
%Desweiteren ist eine enge Zusammenarbeit mit kommunalen Praxispartnern bei der Szenarienerstellung und -bewertung geplant.
Das WI entwickelt und betreibt seine Forschung in steter Zusammenarbeit mit Praxispartner*innen aus Zivilgesellschaft, Politik und Wirtschaft. 


Im Rahmen von Workshops bringen Akteure aus Stadt- und Verkehrsplanung, Kommunalverwaltung, Gebäudeverwaltung, Bildung und Klimaschutzmanagement ihr Praxiswissen ein. Auch in der Phase der Bewertung von Modellierungsergebnissen und Anpassung der Szenarien ein nehmen sie eine wichtige Rolle ein. Welche Praxispartner*innen im Nachwuchsforschungsvorhaben in welcher Weise eingebunden werden hängt von der empirischen und theoretischen Schwerpunktsetzung der Qualifikationsarbeiten ab.
%Die Erstellung der Energieszenarien wird in Zusammenarbeit mit Praxispartnern erfolgen. Das interdisziplinäre Institut der Europa-Universität Flensburg kann hierbei auf transdiziplinäre Forschungs-Erfahrung aus der partizipatorischen Erstellung von Klimaschutzkonzepten (REFERENZ) und dem vom BMBF geförderten Forschungsprojekt „Entwicklungschancen und -hemmnisse einer suffizienzorientierten Stadtentwicklung (EHSS)“ (2017-2020) aufbauen und die bestehenden Kooperationen mit der Stadt und dem Klimapakt Flensburg ausbauen.




%ANMERKUNG UTA:\\
%\textit{Evtl. Vom Text her kürzen und graphisch anreichern (die beteiligten Organisationen in einer Beziehungsgraphik, hierzu hatte ich einen Vorschlag gemacht).}