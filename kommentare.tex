In diesem Abschnitt sollte meiner Ansicht noch thematisiert werden, warum Suffizienzpolitik überhaupt notwendig ist! Ein paar Stichworte:
Apr 23, 2018 2:02 PM

leon.leuser: Lange Zeit wurde Suffizienz individualisiert, aber gesellschaftl. Rahmenbedingungen und Strukturen (Infrastrukturen, Preisstrukturen etc...) prägen das Handeln. Daher ist es relevant diese in den Blick zu nehmen und dahingehend umzugestalten, dass suffizienzorientiertes Handeln zunächst ermöglicht und schließlich bestärkt wird.

Hier könnte noch eine Referenz zu der Diss von Laura Spengler rein, die polit-philosophisch belegt, dass Suffizienzpolitik mit unserem liberalen Rechtsstaat kompatibel ist
Apr 23, 2018 1:58 PM

leon.leuser: http://www.nomos-shop.de/Spengler-Sufficiency-as-Policy/productview.aspx?product=29912