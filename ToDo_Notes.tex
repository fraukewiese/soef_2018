NACH dem Kürzen:
- gendern
- Suffizienzpolitik zusammen geschrieben
- Forschungs-Schwerpunkt statt Teilprojekt
- Wuppertal Institut WI
- sozial oder Sozial-ökologisch (wann was?)
- AP1 oder AP 1



1 Hintergrund und Zielstellung
- Effizienz Zahlen aktualisieren
- Grafik Zusammenspiel: Effizienz dazu
- Update Grafik Szenarien
- lange Sätze?
- Ziele Reihenfolge
- Referenzen einfügen
- Gesamtlesen
- pushen
- in Format bringen


Fragen an alle:
- Titel und Akronym?
- welche Personen fest beschreiben, welche anonym?
- Szenarien 1x auf Deutschland und 1x auf Kommune Flensburg bezogen?

vom Wuppertal Institut:
- Logo
- Adresse auf Titelseite
- CV Mentorin


NEC:
- Logo?


TEXT
- Kurzfassung 2000 Zeichen überarbeiten
- 1 Zielstellung: Anmerkungen einarbeiten, 2.Teil aussformulieren (Frauke) - gegenlesen (Wuppertal)
- 2 Stand von Wissenschaft und Technik schreiben: ESM (FRAUKE), sozial-ökologischer Transformationsforschung und Zusammenhang Energiewende (NEC), Suffizienz-POlitk besonders im Bezug auf Energiewende (Wuppertal)
- 3 Bezug zu sozial-ökologischer Forschung und Förderzielen: Könnte jemand? Link ist vorhanden
- 4 Arbeitsprogramm:
Jeder beschreibt sein Forschungsfeld inklusive Meilensteine
AP2: NEC schreibt
weitere APs: Gerne lesen/ergänzen/umschreiben bei Bedarf
- 5 Kooperation: Ich mache einen Vorschlag
- 6 Betreuungskonzept:
Jeweils 1-2 Sätze zur Betreuung der Doktoranden an NEC und Wuppertal
Wuppertal: bitte Mentor bennenen und Nachweis deren/dessen Expertise in Bezug auf inter- und transdisziplinäre Forschung (im Text nur kurz und dann auf den CV im Anhang verweisen)
- 7 Institutionen und Zukunftsperspektiven
NEC: bitte Bernds Vorschlag in wenigen Sätzen zusammenfassen sowie evtl. ein Satz zu Zukunftsperspektiven und erwartetes Ergebnis, Anwendungspotential, Ergebnisverwertung
Wuppertal: bitte ein kurzen Abschnitt zur Institution sowie evtl. ein Satz zu Zukunftsperspektiven und erwartetes Ergebnis, Anwendungspotential, Ergebnisverwertung
Wuppertal: Referenzen die zum Projekt passen könnten in den Anhang wenn ihr sowas zur Hand habt, ich denke es ist nicht unbedingt verlangt
konkrete Maßnahmen der Öffentlichkeitsarbeit und des Wissenstransfers würde ich versuchen zu ergänzen, aber wenn ihr Ideen habt, schreibt es gern dazu!


- Referenzen: entweder direkt ins bib-file und mit \cite{}  oder die Referenz in Klammern dahinter schreiben (doi)
- rechtsklick, dann kommentieren.