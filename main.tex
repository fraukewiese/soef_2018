
\documentclass[a4paper,11pt,twoside]{scrartcl}
\usepackage[utf8x]{inputenc}
\usepackage{graphicx}
\usepackage{geometry}
\usepackage[ngerman]{babel}
%\usepackage{babelbib}
%\usepackage[backend=biber]{biblatex}
\usepackage{units}
\usepackage{url}
\usepackage{setspace}
\usepackage[
	pdftitle={Energie_Suffizienz},
 	pdfauthor={et.al.},
% 	pdfsubject={},
% 	pdfkeywords={},
	pdfstartview=FitH, % Auf Seitenbreite anpassen (Anzeige)
	pdfborder={0 0 0},
%  	bookmarks=true,
% %	plainpages=false,
	colorlinks=false,
	hyperfootnotes=false,
	pagebackref=false]{}
\usepackage[colorinlistoftodos,prependcaption,textsize=normalsize]{todonotes}  % disable
\usepackage{amsmath}
\usepackage{amstext}
\usepackage{amssymb}
\usepackage{color}
\usepackage[numbers]{natbib}
\usepackage{pdfpages}
\usepackage{hyphenat}
\usepackage{pdflscape} % für drucken ändern auf package lscape /pdflscape
\usepackage{textpos}
\usepackage{microtype}
\usepackage{enumitem}
\usepackage{multirow}
\usepackage{pdfcomment}

\usepackage{pgfgantt} % gantchart für den Arbeitsplan

% \usepackage{hyphenat}
\usepackage{float}
\usepackage{placeins}
\RequirePackage[bf]{caption}

\renewcommand{\textfraction}{.01} % vorher: .2
\renewcommand{\floatpagefraction}{.99}% vorher: .5
\renewcommand{\topfraction}{0.9}	% max fraction of floats at top
\renewcommand{\bottomfraction}{0.9}	% max fraction of floats at bottom

\newcommand{\ltab}{\raggedleft\arraybackslash}
\newcommand{\ctab}{\centering\arraybackslash} 
\newcommand{\rtab}{\raggedright\arraybackslash}

\usepackage{tabularx}
\newcolumntype{L}[1]{>{\raggedright\arraybackslash}p{#1}} % linksbündig mit Breitenangabe
\newcolumntype{C}[1]{>{\centering\arraybackslash}p{#1}} % zentriert mit Breitenangabe
\newcolumntype{R}[1]{>{\raggedleft\arraybackslash}p{#1}} % rechtsbündig mit Breitenangabe

\newcommand{\rem}[1]{}

% \renewcommand{\figurename}{Abb.}
% \renewcommand{\tablename}{Tab.}

\usepackage{acronym}
%\renewcommand{\bflabel}[1]{\normalfont{\normalsize{#1}}\hfill}

\usepackage[automark]{scrpage2}
\pagestyle{scrheadings}
\clearscrheadfoot
\ohead{\headmark}
\ofoot{\pagemark}
\setheadsepline{0.4pt}
\setfootsepline{0.4pt}

\setkomafont{pageheadfoot}{\rmfamily\small}

\renewcommand{\thefigure}{\arabic{section}-\arabic{figure}}
\renewcommand{\thetable}{\arabic{section}-\arabic{table}}

\newcommand{\entspricht}{\mathrel{\widehat{=}}}

\geometry{left=25mm,right=25mm, top=28mm, bottom=28mm}
\parindent 0pt
\parskip 11pt

% kein Platz zwischen \items
\setlist{nosep}
% Platz nach Überschriften reduzieren
\usepackage{titlesec}
\titlespacing*{\section}{0pt}{0pt}{0pt}
\titlespacing*{\subsection}{0pt}{0pt}{0pt}
\titlespacing*{\subsubsection}{0pt}{0pt}{0pt}
\titlespacing*{\paragraph}{0pt}{0pt}{5pt} % horizontal spacing in paragraph

\begin{document}
\onehalfspacing

\clearpage


{\singlespacing

\thispagestyle{empty}
\begin{center}

%first row logos
\begin{figure}[htb]
    \centering
    \begin{minipage}[c]{0.3\linewidth}
        \centering
        \includegraphics[width=5cm]{logos/WI_Logo_CMYK.pdf}
    \end{minipage}
    \hfill
    \begin{minipage}[c]{0.35\linewidth}
         %\centering
         \includegraphics[width=6cm]{logos/AbtEUM.JPG}
     \end{minipage}
    \hfill
    \begin{minipage}[l]{0.25\linewidth}
        %\centering
        \includegraphics[width=4.5cm]{logos/NEC_logo.jpg}
    \end{minipage}
\end{figure}

\iffalse
%second row logos
\begin{figure}[htb]
    \centering
    \begin{minipage}[c]{0.3\linewidth}
        \centering
        \includegraphics[width=2.2cm]{logos/2015_Logo_TUM_RGB.jpg}
    \end{minipage}
    \hfill
    \begin{minipage}[c]{0.35\linewidth}
        \centering
        \includegraphics[width=5.5cm]{logos/isea_rwth_logo.jpg}
    \end{minipage}
    \hfill
    \begin{minipage}[c]{0.3\linewidth}
        \centering
        \includegraphics[width=3cm]{logos/TU_Logo_lang_RGB_rot.png}
    \end{minipage}
\end{figure}
\fi
\vspace*{1.5 cm}

{\LARGE\textbf{\textsf{Skizze Nachwuchsforschungsgruppe}}

\textsf{\textit{zur Bekanntmachung \glqq inter- und transdisziplinär arbeitende\\ Nachwuchsgruppen im Rahmen der Sozial-ökologischen Forschung\grqq} }
}

\vspace{0.5cm}

{\Huge
\textbf{\textsf{Die Rolle von Energie-Suffizienz in Energiewende und Gesellschaft\\
%oder\\
%Die Rolle von Energie-Suffizienz bei der Transformation zu einem nachhaltigen Energiesystem (RESTNE)\\
%oder\\
%Die Rolle von Energie-Suffizienz in der Energiewende
}}

\textbf{\textsf{}}
}

{\LARGE
\textbf{\textsf{Akronym:{EnSu}}}
}

\vspace{1.5cm}

\textit{Eingereicht durch die Verbundkoordinatorin}

{\parskip 0pt
Frauke Wiese\\
i$^{2}$ Interdisziplinäres Institut für Umwelt-, Human- und Sozialwissenschaften\\
Europa-Universität Flensburg, Auf dem Campus 1, 24943 Flensburg}

\vspace{0.3cm}

\textit{und den Verbundpartner}

{\parskip 0pt
Benjamin Best\\
Wuppertal Institut für Klima, Umwelt, Energie gGmbH\\
Döppersberg 19, 42103 Wuppertal Wuppertal}

\vspace{0.5cm}

An den Projektträger im DLR \\
AE 41 Globaler Wandel/Klima- und Umweltschutz, Sozial-ökologische Forschung \\
Heinrich-Konen-Straße 1, 53227 Bonn


\end{center}


\clearpage
}

\setcounter{page}{1}

\section{Zielstellung und gesellschaftlicher Bedarf}
\label{sec:ziel}
%\textit{Beschreibung der Problem- und Zielstellung sowie des gesellschaftlichen Bedarfs}

Klimawandel, Ressourcenverknappund und zunehmende Umweltveränderungen lassen die Transformation des Energiesystems zu einer der der zentralen gesellschaftlichen Herausforderungen werden. Noch ist weitgehend offen, wie die Versorgung mit Strom, Wärme, Mobilität sichergestellt werden kann, während Klimaziele erreicht, soziale Gerechtigkeit gewahrt und die natürlichen Grenzen unseres Planeten langfristig eingehalten werden können. 

Energiesystem-Modelle haben sich als Werkzeuge etabliert, um technisch mögliche und ökonomisch vorteilhafte Energiewende-Pfade darzustellen und zu analysieren. Sie helfen dabei, komplexe Zusammenhänge zwischen technischen Möglichkeiten (z.B. Flexibilität), Umweltbedingungen (z.B. Wetter-abhängige Erneuerbare), Marktregeln (z.B. Energy-only-Markt) im Licht klimapolitischer Zielsetzungen zu verstehen und unterstützen somit die Klima- und Energiepolitik.

Ein entscheidender Einflussparameter für berechnete Szenarien ist die zukünftige Nachfrage nach Strom, Wärme und Mobilität. Die Mehrzahl der bisherigen Modelle legt den Fokus  auf die Bereitstellung von Energiedienstleistungen. Die Nachfrage-Seite wird meist nur innerhalb enger Grenzen variiert. So berücksichtigen viele Modelle beispielsweise die Verschiebung der Nachfrage von fossilen Brennstoffen zu Strom durch die Elektifizierung im Wärme- und Transportsektor. Eine übergreifende Perspektive, welche die verschiedenen Bereiche berücksichtigt und integriert, in denen Energie gebraucht wird, steht noch aus. Zudem wird in den meisten Modellen die absolute Nachfrage nach Energie nicht in Frage gestellt. Mit anderen Worten: Bislang sind Modellierungen weitgehend blind gegenüber Veränderungen der Nachfrage nach Energie durch gesellschaftlichen Wandel oder Verhaltensänderungen.

Nicht nur in der Modellierung von Energiewende-Pfaden, sondern auch  bei Klimaschutz-Strategien liegt der Fokus bisher auf im weitesten Sinn technikorientierten Lösungen bei der Erzeugung und Distribution von Energie für Wohnen, Bauen, Ernährung und Mobilität \cite{Creutzig2018}. 

Zwar bieten laut Modellrechnungen Erneuerbare Energien und Effizienz die Möglichkeit im Jahr 2050 80-95 Prozent Treibhausgas-Reduktion in Deutschland \cite{BMWi2017} zu erreichen. Es gibt jedoch verschiedene Gründe keine der drei Nachhaltigkeitsstrategien Konsistenz, (Erneuerbare Energien ersetzen Fossile), Effizienz (relative Reduktion des Energieverbrauchs bei Bereitstellung der gleichen Energiedienstleistung) und Suffizienz (absolute Reduktion der Nachfrage nach Energiedienstleistungen durch veränderte soziale Praktiken und gesellschaftliche Leitbilder) außer Acht zu lassen: Zum einen setzten die ambitionierten Ziele des Pariser Klimaabkommens die Herausforderung nochmal auf eine neue Stufe \cite{Rogelj2018}. Außerdem sprechen Aspekte wie Flächenverbrauch (Referenz fehlt noch), Ressourcenbedarf (Referenz fehlt noch) und Akzeptanzfragen \cite{Fuchs2016}, zusammengefasst die Einhaltung der planetaren Grenzen \cite{Rockstroem2009} dafür, die Chancen die Suffizienz bietet, nicht zu übersehen \cite{SAMADI2017}.

Desweiteren ist bisher nicht abschließend geklärt, in welchem Ausmaß Emissionsreduktionen in Deutschland durch Konsistenz und Effizienz tatsächlich erreicht wurden. Verlagerungseffekte in andere Länder könnten ebenso eine Rolle gespielt haben \cite{Wiedmann2015}, so dass die Emissionsreduktionen bei Bilanzierung nach Verursacherprinzip geringer ausfallen. Auch die Rolle, die Effizienz spielen kann, ist nicht klar. Zielszenarien für Deutschland gehen von einer Reduzierung der Stromnachfrage durch eine massive Steigerung von Effizienz aus \cite{BMWi2017}. Zugleich findet jedoch in der fachwissenschaftlichen und energiepolitischen Diskussion der Befund Aufmerksamkeit, dass Energieeffizienzsteigerungen in der Regel von Rebound-Effekten begleitet werden, welche die Einsparungen teilweise kompensieren oder mitunter sogar zu einem verstärkten Energieverbrauch („Backfire“) führen \cite{DeutscherBundestag2013,Santarius2012}. Auch aufgrund solcher Rebound-Effekte wurden in den vergangenen Jahren die Energiesparpotenziale nicht voll ausgeschöpft, was dazu beitrug, dass in Deutschland trotz aller Maßnahmen zur Verbesserung der Energieeffizienz der Stromverbrauch im Jahr 2016 gegenüber 2008 (acht Jahre) um nur 1,5 Prozent gesunken ist \cite{UBA2017}. Dies lässt das erklärte Ziel weitere 8,5 Prozent bis 2020 (vier Jahre) zu schaffen sehr ambitioniert erscheinen.

In Anbetracht des möglichen Beitrags der Suffizienz und der Größe der Herausforderung, ist die Rolle von Energie-Suffizienz derzeit unterrepräsentiert in Diskussion und Forschung für Klimaschutz und Energiewende. Abbildung \ref{fig:zusammenspiel} verdeutlicht schematisch den Gedanken des Zusammenspiels der drei Dimensionen für die Erreichung der Klimaziele. 


\begin{figure}[!h]
    \centering
    \includegraphics[width=0.8\textwidth]{figures/Zusammenspiel2.pdf}
    \caption{Schamatisch: Bisherige Emissionsreduktion durch Konsistenz (links), notwendiger Beitrag von Effizienz und Suffizienz in Zukunft (rechts) zur Erreichung der Klimaziele in Deutschland}
    \label{fig:zusammenspiel}
\end{figure}

Weiter hat sich in der Forschung die Erkenntnis durchgesetzt, dass nur unter Einbeziehung und dem Zusammenspiel verschiedener Disziplinen - wie technisch-/ingenieurwissenschaftlicher Fachrichtungen, Politik-, Wirtschafts- und Sozialwissenschaften - die gesellschaftliche Herausforderung der Transformation des Energiesystems erfolgreich bewältigen lässt \cite{WGBU2012}.

Besonders deutlich wird die Notwendigkeit der verstärkten interdisziplinären Zusammenarbeit bei der Erstellung von Energie-Szenarien. Gesellschaftliche Entwicklungen haben enormen Einfluss auf die Nachfrage nach Energie-Dienstleistungen, die politischen Rahmenbedingungen für die Erzeugungs- und Distributionswege sowie die technisch-ökonomischen Optionen zur Erfüllung der Energienachfrage (schematisch dargestellt in Abbildung \ref{fig:szenarien}). Gängige Praxis in der Energiesystem-Modellierung ist es, nur Letzteres in den Blick zu nehmen, gesellschaftliche Dynamiken hingegen weitgehend zu ignorieren.  Dieses Vorgehen erlaubt eine Abschätzung der Leitungsfähigkeit sozio-technischer Innovationen. Allerdings um den Preis einer Vorstellung, die Gesellschaft als mehr oder weniger statisches Konstrukt imaginiert und nicht als das hoch dynamische, beständigen Veränderungen unterworfene Gemeinwesen, das sie ist.   (Abbildung \ref{fig:szenarien} links). Für konsistente und damit belastbare Energie-Szenarien ist die Integration sozialwissenschaftlicher Wissensbestände in Bezug auf künftig zu erwartende gesellschaftliche Transformationsprozesse und ihre Auswirkungen auf den gesellschaftlichen Metabolismus unverzichtbar. Beispielhaft seien hier die Digitalisierung oder der demografische Wandel genannt. Beide Prozesse werden mit hoher Wahrscheinlichkeit Auswirkungen auf die Nachfrage nach Energiedienstleitungen haben, in welchem Umfang und in welche Richtung - steigende oder sinkende Nachfrage - ist erstens kontingent und zweitens allein mit Verweis auf die Leistungsfähigkeit technischer Innovationen nicht zu beantworten. Noch nicht abzusehen ist auch, in welchem Ausmaß etwa kommunale und nationale Klimaschutzmaßnahmen - erinnert sei hier an die aktuelle Diskussion um den ticketlosen ÖPNV - soziale Innovationen oder einen Wandel gesellschaftlicher Werte und Normen beeinflussen werden. (Abbildung \ref{fig:szenarien} rechts). Außerdem ermöglicht die Erweiterung des Bezugsrahmen um die gesellschaftliche Perspektive, Suffizienz neben Effizienz und Konsistenz in der Modellierung abzubilden. 

\begin{figure}[!h]
    \centering
    \includegraphics[width=0.8\textwidth]{figures/Szenarien.pdf}
    \caption{Rahmen für Energie-Szenarien: Links: Getrennte Betrachtung / Rechts: Einbettung der Annahmen für Energie-Nachfrage, Politikrahmen und technische Optionen in gesellschaftliche Entwicklungen}
    \label{fig:szenarien}
\end{figure}

Ziele der Nachwuchsforschungs-Gruppe Energie-Suffizienz sind:
\begin{itemize}
 \item Einen in der bisherigen Diskussion um die Transformation des Energiesystems in Richtung Nachhaltigkeit unterrepräsentierten Aspekt  -- Suffizienz -- intensiv dahingehend zu erforschen
  \item In der interdisziplinäre Kooperation ein Verfahren für die Erstellung von konsistenten, multiperspektivischen  Energieszenarien zu entwickeln und zu erproben
  \item (Weiter)-Entwicklung eines Energiesystem-Modells, in dem Konsistenz, Effizienz und Suffizienz integriert betrachtet werden können
 \item Einen möglichen und eventuell notwendigen Beitrag von Energie-Suffizienz zur Erreichung der Klimaziele auf nationaler (Deutschland) und kommunaler (Flensburg) Ebene ermitteln
 \item Die Entwicklung von Zukunftsszenarien zur Transformation des Energiesystems in Richtung Nachhaltigkeit 
 \item Die historische Rekonstruktion der Wechselwirkung zwischen Energiesystemtransformation und gesellschaftlichem Wandel  
 \item Handlungsoptionen für Suffizienz-Politik von kommunaler bis EU-Ebene
 \item Literatur und Methoden der Energiesystem-Analyse als auch die der sozial-ökologischen Transformationsforschung erweitern. Schwerpunkt liegt dabei auf Synergien zwischen den Forschungsgebieten.
 \item Das Methodenspektrum junger Forscher mit Hintergründen aus Sozialwissenschaft (sozial-ökologische Transformationsforschung), Wirtschaftingenieurwesen (Energiesystem-Analyse) und Politikwissenschaft interdisziplinär erweitern
\end{itemize}

Im Rahmen des Projektes erarbeitete Daten und Software, werden open source zur Verfügung gestellt. Der Open Science Ansatz der der Nachwuchs-Forschunggruppe zugrunde liegen wird, beinhaltet auch den offenen Zugang zu Publikationen etc. Die Erkenntnisse sollen außerdem politische Entscheidungsprozesse zu Klimaschutz und Energiewende unterstützen und Empfehlungen für Suffizienzpolitik beinhalten. Ein Schwerpunkt wird die Ergebnis-Kommunikation auch im populär-wissenschaftlichen Bereich, um auch eine außerwissenschaftliche Öffentlichkeit zu erreichen.

\section{Stand von Wissenschaft und Technik sowie eigene Vorarbeiten}
\subsection*{Energiesystem-Analyse und Modellierung}
\begin{itemize}
    \item Energiesystem-Analyse und Modellierung: wichtige Rolle auch in policy advice, hochkomplexe Modelle, inzwischen auch Open Source Modelle, wenn auch noch Mangel an Transparenz
    \item Generelle Herausforderungen ESM:
    \begin{itemize}
     \item Open heißt nicht unbedingt verständlich, bisher Mangel an Ergebniskommunikation 
     \item Sozialwissenschaftlicher Komponenten bisher unterrepräsentiert wenn es auch auf dem Gebiet Akzeptanz
     \item Arbeiten im Bereich Energie-Nachfrage beziehen sich viel auf Flexibilität der Nachfrage, weniger auf die absolute Nachfrage nach Strom, Wärme, Mobilität
    \end{itemize}
    \item Eigene Vorarbeiten (Frauke): ESM, open ESM, kritische Betrachtung ESM, Verbindung über techn.-ökono. Sichtweise hinaus, bereits interdisziplinär gearbeitet / (EUM-Team): oemof, open, Klimaschutz (transdisziplinär)
\end{itemize}

\subsection*{Sozial-ökologische Transformationsforschung}
\textit{NEC: bitte ergänzen, v.a. in Bezug auf Energiewende und Suffizienz inklusive eigene Vorarbeiten}
% Rebound text von Bernd, siehe auch Bezug zu Nachwuchsgruppe Nachwuchsforschungsgruppe Rebound, Suffizienz und Digitalisierung)
    %https://www.fona.de/de/nachwuchsfoerderung-sozial-oekologische-forschung-20620.html
    %  1.05.2016 - 30.04.2021
    % Digitalisierung und sozial-ökologische Transformation. Rebound-Risiken und Suffizienz-Chancen digitaler Dienstleistungen
    
\subsection*{Sufizienz-Politiken}
\textit{Wuppertal: bitte ergänzen, v.a. in Bezug auf Energie-Suffizienz inklusive eigene Vorarbeiten}


\section{Bezug zur Sozial-ökologischen Forschung und zu den Förderzielen}

\url{https://www.fona.de/mediathek/pdf/SOEF_Foerderkonzept_barrierefrei.pdf}
% Bezug zu Sozial-ökologischer Forschung
Es ist erklärtes Ziel der Sozial-ökologischen Forschung System-, Orientierungs- und Entscheidungswissen zu den zentralen Nachhaltigkeitsherausforderungen bereitzustellen. Die Nachwuchsforschungsgruppe stellt solches Wissen bezüglich der gesellschaftlichen Herausforderungen Energiewende und Klimaschutz bereit. Sie analysiert den Transformationsbedarf in Wirtschaft und Gesellschaft mit Fokus auf die bisher unterrepräsentierte Option der Energie-Suffizienz. Damit macht sie ein Lösungsvorschlag zu der ökologsichen Krise des Klimaschutzes: Erwweiterung der Energiewende-Optionen um Suffizienz und sich stärker der Praxistauglichkeit annähern. Wie in der Sozial-ökologischen Forschung üblich, werden Analysen durchgeführt um die Wechselwirkungen zwischen Gesellschaft, Wirtschaft und um Umwelt abzuschätzen, in diesem Fall die Wechselwirkung von Suffizienz-Politiken, Lebensstilen und der Nachfrage nach Energiedienstleistungen. Fokus wird dabei besonders auf gesellschaftliche Entwicklungen und deren Auswirkingen aufs Energiesystem (v.a. Nachfrage) gelegt. Gemäß des Forschungsansatzes Sozial-ökologischer Forschung wirken hierfür unterschiedliche Disziplinen zusammen und werden ergänzt durch Wissen aus der Praxis. Die erwarteten Handlungsemfpehlungen zu Energiesuffizienz die am Endes des Projektes stehen sollen realistische Lösungsoptionen sein für den Übergang zu einer nachhaltigen Gesellschaft und Energiesystem, die Praxisnähe wird durch die gezielte Beteiligung einer Beispielkommune erreicht.

Thematisch widmet sich das Projekt mit Klimaschutz und Energiewende zwei in der Sozial-ökologischen Forschung als zentral bezeichnete Nachhaltigkeitstransformationen. Methodisch wird für die Szenarienerstellung eine Methode weiterentwicklet die die Integration des Wissens der Disziplinen .... erleichtern soll.
Im thematischen Schwerpunkt Nachhaltiges Wirtschaften wird unter der Rebound-Thematik die Erabeitung neuer ANsätze zur Ressourcenschonung genannt als dringend notwendie erachttet. Dies wird mit der Suffizienz-Thematik aufgegriffen.
Vor allem thematisch verortet in einem relativ neuen Schwerpunkt Sozial-ökologischer Forschung in Zusammenarbeit und Vernetzung mit anderen Bereichen. Schwerpunkt im FONA-Rahmenprogramm. Nach dem Vorbild der Fördermaßnahme Umwelt- und gesellschaftsverträgliche Transformation des Energiesystems. Da passt es genau rein und hat Bezug. 


SEHR GERN ERGÄNZEN

% Bezug zu sozial-ökologischer Forschung
%Ideen:
%interdisziplinäres Verständnis wird geschaffen und baut auf dem disziplinären Vorwissen auf indem teilweise gleiche Arbeitspaket erst disziplinär parallel ausgeführt werden um dann in der Synthesephase den gemeinsamen Blick drauf zu werden (z.B. Szenarien-Methoden)
%...gemeinsamen Methodenentwicklung vor dem Hintergrund transdisziplinärer Nachhaltigkeitsforschung....
% Kultur des interdisziplinären Veröffentlichen pflegen
%%%%%%
% Versorgungssicherung mit nachhaltiger Energie
% gute Lebensqualität in Zukunft sicher stellen
% Umweltprobleme in Verbindung mit den wirschaftlichen, sozialen und politischen Strukturen die sie verursachen in Verbindung bringen
% wirtschaftliche soiale und ökologische Belange gleichermaßen berücksichtigen

% Bezug zu Förderzielen
Die Weiterentwicklung institutioneller Kapazitäten zur Durchführung transdisziplinärer Nachhaltigkeitsforschung wird durch die Verankerung der Nachwuchsgruppe am Interdisziplinären Institut für Umwelt-, Sozial und Humanwissenschaften an der Europa-Universität Flensburg geschaffen. In der eigenverantwortlichen Arbeitsgruppe bekommen junge Wissenschaftler*innen die Möglichkeit sich auf der Basis ihres disziplinären Vorwissens sozial-ökologischen Fragestellungen zu widmen. Möglichkeiten zur Weiterqualifizierung werden durch die  Nachwuchsgruppe für wissenschaftliche Mitarbeiter*innen geschaffen, die zwar schon jetzt an den Schnittstellen forschen, dafür aber nicht den entsprechenden Rahmen haben um sich auch wissenschaftlich zu qualifizieren.


\section{Forschungsarbeit, Arbeitsprogramm, Methoden, Disziplinen}
\textit{Beschreibung der geplanten Forschungsarbeiten und des Arbeitsprogramms, unter Einschluss der Darstellung von Methoden, die zur Anwendung kommen bzw. entwickelt werden sollen; sowie der disziplinären Zusammensetzung der geplanten Nachwuchsgruppe}

%%%%%%%%%%%

\begin{figure}[!h]
    \centering
    \includegraphics[width=0.8\textwidth]{figures/Forschungsarbeit.pdf}
    \caption{Forschungsprogramm: Disziplinäre Forschungsarbeit (vertikal) und  interdisziplinäre (horizontal) Arbeitspakete im Verlauf der fünf Jahre sowie geplanter disziplinärer und interdisziplinärer Output}
    \label{fig:forschungsprogramm}
\end{figure}

Abbildung \ref{fig:forschungsprogramm} gibt einen Überblick wie die disziplinären Forschungsfelder (vertikal dargestellt) im Verlauf der fünf Jahre über die interdisziplinären Arbeitspakete (horizontal dargestellt) ineinander greifen. Im Folgenden wird jedes Forschungsfeld mit Meilensteinen sowie die interdisziplinären Arbeitspakete kurz erläutert.

\subsection*{Forschungsfeld sozial-ökologische Transformationsforschung: Gesellschaftlicher Wandel und Energieverbrauch}
\textit{NEC: Bitte ergänzen}
Der Zusammenhang zwischen gesellschaftlichem Wandel und Energieverbrauch soll zuerst historisch untersucht werden, um dann zukünftige Entwicklungen zu entwerfen. Mit Methoden des Transformationsdesigns wird dann ...

\textbf{Meilensteine}
T1: Zusammenhang gesellschaftlicher Wandel und Energieverbrauch - Vergangenheit und Gegenwart\\
T2: Zusammenhang gesellschaftlicher Wandel und Energieverbrauch - Zukunft\\
T3: Szenarien-Methode\\
...\\
T Output: Zukunftserzählungen von gesellschaftlichen Suffizienz-Pfaden

\subsection*{Forschungsfeld Politikwissenschaft: Suffizienz-Politiken und Rahmenbedingungen}
\textit{Wuppertal: Bitte umformulieren/ergänzen}
\textbf{Meilensteine}
Ö1: Regulatorische und politische Rahmenbedingungen für Konsistenz, Effizienz und Suffizienz - Vergangenheit und Gegenwart\\
Ö2: Regulatorische und politische Rahmenbedingungen für Konsistenz, Effizienz und Suffizienz - Zukunft\\
Ö3: Szenarien-Methode\\
Ö4: Quantifizierung der Auswirkungen (Emissionsreduktion, Kosten) verschiedener regulatorischer Maßnahmen zu Konsistenz, Effizienz und Suffizienz\\
Ö5: \\
Ö Output: Zusammenfassung Konsistenz, Effizienz und Suffizienz Maßnahmen auf individueller, kommunaler, nationaler und EU-Ebene

\subsection*{Forschungsfeld Systemanalyse: Integrierte Energie-System-Modellierung: Beitrag von Konsistenz, Effizienz und Suffizienz zur Energiewende}

\textbf{Meilensteine}
S1: Auswahl eines offenen Energiesystem-Modells (erweiterbar, anpassbar, open source, gewisser Nutzerkreis)\\
S2: Erweiterung ESM um die ermittelten Nachfrage-Parameter\\
S3: Szenarien-Methode\\
S4: Modellierung: Quantifizierung des Beitrags von Konsistenz, Effizienz und Suffizienz zu einem Energiesystem mit geringem Anteil fossiler Brennstoffen\\
S5: Ergebnis-Kommunikation\\
S Output: Integriertes ESM für Konsistenz, Effizienz, Suffizienz wird unter einer offenen Lizenz Wissenschaft und Gesellschaft zur Verfügung gestellt

\subsection*{AP1: Interdisziplinärer Forschungsrahmen}
Am Beginn der Forschungsgruppe steht der gemeinsame Entwurf eines interdisziplinären Bezugsrahmens für Forschung zu Energie-Suffizienz aus Sicht der Energiesystemanalyse, des Transformationsdesigns und aus ökonomischer/regulatorischer Sicht. Neben dem inhaltlichen "aufeinander einlassen" werden auch die Formen und Formate für die gemeinsame Zusammenarbeit entworfen, sowohl inhaltliche, also auch organisatorische wie z.B. Häufigkeit der Treffen etc.

\subsection*{AP2: Gesellschaftliche Indikatoren für Energieverbrauch}
\textit{NEC liefert Textbaustein}
Um Suffizienz im Energiesystem-Model abbilden zu können bedarf es einer Übersetzung von gesellschaftlichen Indikatoren zu quantifizierbarem Energieverbrauch. Dieser sozioligisch-technisch-ökonomischen Schnittstelle widmet sich dieses AP. Das Forschungsfeld Transformationsdesign ermittlet relevante gesellschaftliche Einflussfaktoren auf die Energienachfrage. Diese bilden die Grundlage für die Auswahl der Indikatoren, die in das Energiesystem-Modell als Input-Parameter aufgenommen werden. Ein Beispiel zur Verdeutlichung: Das gestiegene Durchschnittsalter der Bevölkerung führt zu gesteigerter Wohnfläche, was sich wiederum in Wärmebedarf ausdrücken lässt. Oder: Eine Reduktion der durchschnittlichen Arbeitszeit führt zu einem erhöhten Mobilitätsaufkommen (BESSERE BEISPIELE EINFÜGEN). Im Forschungsfeld Ökonomie wurden relevante regulatorische Einfussfaktoren auf Suffizienz, Konsistenz und Effizienz ermittelt sowie mögliche zukünftige Politik-Maßnahmen erdacht. In interdisziplinärer Zusammenarbeit wird die Übersetzung ins Energie-System-Modell methodisch erarbeitet. Diese bildet die Grundlage für die Erweiterung des Energiesystem-Modells.
%Das kann zum Beispiel die EnEV sein, die als Effizienz-Maßnahme zu verstehen ist, da sie nicht auf den Wärmebedarf pro Person sondern auf den Wärmebedarf pro Wohnfläche abzielt. 

\subsection*{AP3: Szenarien-Methode}
Dieses methodische Arbeitspaket wird aus disziplinärer Sicht vorbereitet: Verschiedene Methoden zur Szenarienerstellung werden auf ihre Eignung für interdisziplinäre Szenarien geprüft die alle Kompenenten aus Abbildung \ref{fig:szenarien} integriert betrachten. Ein Auswahlkriterium soll hierbei die Eignung für die Beteiligungung von Praxispartnern in der Szenarienerstellung sein. Im Anschluss an die disziplinäre Vorarbeit werden wird in interdisziplinärer Zusammenarbeit eine passende Methode zur Szenarien-Erstellung ausgewählt oder bestehende disziplinäre kombiniert und wo notwendig erweitert. MÖGLICHE METHODEN NENNEN (e.g. Cross-Impact-Balance .....)

\subsection*{AP4:Szenarien-Erstellung}
Im Anschluss an die Auswahl der Methode werden in dieser inter- und transdisziplinären Arbeitsphase konsistente Gesellschafts-Energiesystem-Szenarien erstellt. Im Rahmen von Workshops und Befragungen werden Praxipartner in die Szenarienerstellung einbezogen. SZENARIO-FOKUS FESTLEGEN: z.B. SZENARIEN FÜR FLENSBURG (kommunal) UND SZENARIEN FÜR DEUTSCHLAND?

\subsection*{AP5:Iteration Szenarien}
Die konkrete Übersetzung der Szenarien in quantifizierte Daten als Modell-Input macht mindestens eine Iterationsschleife und eine Anpassung oder Ergänzung der Szenarien erforderlich. Die Ergebnisse werden interdisziplinär verifiziert und validiert (METHODE NENNEN/REFERENZIEREN?) und im Anschluss wo notwendig berichtigt, erweitert und angepasst.

\subsection*{AP6:Synthese und Ausblick}
Im letzten Teil des Projektes steht die Synthese-Phase der Ergebnis, die als Ergänzung zu disziplinären Publikationen einen Fokus auf die die Publikation von interdisziplinären Methoden, Erfahung und praktische Verwertbarkeit der Ergebnisse legt. Dies geht Hand in Hand mit einer Ausblicks-Phase, in der Nachwuchs-Forscher*innen sich mit weiteren Qualifizierungs- und interdisziplinären Projektmöglichkeiten beschäftigen

\section{Kooperationen, Forschungs- und Praxispartner}
\textit{vorgesehene Kooperationen (Forschungs- und Praxispartner) und Arbeitsteilung, Einbindung der Praxispartner in den transdisziplinären Forschungsansatz}\\

% Allgemein EUF EUM NEC Wuppertal
% Frage: welche Abteilung im Wuppertal Institut?
Die Nachwuchsgruppe ist am interdisziplinären Institut für Umwelt-, Sozial- und Humanwissenschaften der Europa-Universität Flensburg (EUF) und dem Wuppertal Institut für Klima, Umwelt, Energie angesiedelt. Das Forschungsfeld der sozial-ökologischen Transformationsforschung der Nachwuchsgruppe iist am Norbert Elias Center für Transformationsdesign und Forschung verortet während die Abteilung Energie- und Umweltmanagement den Teil der Energie-System-Analyse übernimmt. Beide Abteilungen sind Teil des interdisziplinären Instituts an der EUF und verstärken damit ihre Zusammenarbeit auch innerhalb des Instituts. Am Wuppertal Institut wird das Forschungsfeld Suffizienz-Politk bearbeitet. Die interdisziplinären Arbeitspakete werden in enger Kooperation zwischen der Universität und der außeruniversitären Forschungseinrichtung durchgeführt. In mindestens halbjährlich stattfindenden Treffen aller Mitarbeiter*innen der Forschungsgruppe werden Forschritte diskutiert und Arbeitsschritte geplant, unterstützt von der Mentorin. Diese Gesamttreffen werden flankiert von weiteren bilateralen Arbeitstreffen und intensiven Phasen der interdisziplinären Forschungsarbeit durch mehr-monatige Forschungsaufenthalte in der Partner-Institution durch die Doktoranden. Der genauere Rahmen für die Zusammenarbeit wird im ersten Jahr der Konstituierung der Gruppe gemeinsam festgelegt.

% Leitung geteilt
Zur Verstärkung des interdisziplinären Charakters wird die Forschungsgruppe zu gleichen Teilen von der EUF und dem Wuppertal Institut geleitet (jeweils eine 0,75-Stelle). Dies trägt auch zur engeren Kooperation zwischen Universität und außeruniversitärer Forschungseinrichtung bei.

% EUM Kooperationen
Der Bereich Energiesystemanalyse wird bereits bestehende Forschungskooperationen mit der Open Energy Modelling Initiative \cite{}
einbringen. Da hierin so gut wie alle in Deutschland ansässigen Institutionen, die offene Energiesystem-Modellen betreiben, vertreten sind, ist dies besonders hilfreich für die Wahl und Nutzung eines der bestehenden Energiesystem-Modelle. Ob das von der Abteilung EUM in Kooperation mit der Zentrum für Nachhaltige Energiesysteme und dem Reiner Lemoine Institut erarbeitete Energiesystemmodellierungs-Framework 'oemof' in der Nachwuchsgruppe genutzt und erweitert wird, bleibt einer genauen Prüfung und dem Vergleich mit anderen Optionen vorenthalten. 

% NEC Kooperationen?
% Wuppertal Kooperationen?

% Praxispartner
Desweiteren ist eine enge Zusammenarbeit mit den Praxispartnern Stadt Flensburg und Klimapakt Flensburg bei der Szenarienerstellung und -bewertung geplant. Im Rahmen von Workshops bringen Akteure aus Stadt- und Verkehrsplanung, Kommunalverwaltung, Gebäudeverwaltung, Bildung und Klimaschutzmanagement ihr Praxiswissen ein. Auch in der Phase der Bewertung von Modellierungsergebnissen und Anpassung der Szenarien ein nehmen sie eine wichtige Rolle ein. Ein Letter of Intent ist dem Anhang beigefügt.


\section{Betreuungskonzept}
\textit{(inklusive der/des vorgesehenen Mentorin bzw. Mentors und Nachweis deren/dessen Expertise in Bezug auf inter- und transdisziplinäre Forschung}

\textit{NEC: 1-2 Sätze Betreuungskonzept am NEC}

\textit{Wuppertal: 1-2 Sätze Betreuungskonzept WI und Nachweis Expertise in Bezug auf inter- und transdisziplinäre Forschung der Mentorin.}

Die Gruppenleiterin Frauke Wiese betreut die Promotion in der Energiesystemanalyse, der Gruppenleiter des Wuppertal Instituts im Bereich Suffizien-Politiken, während die die disziplinäre Betreuung der beiden Promotionen im Bereich sozial-ökologische Transformationsforschung am Norbert Elias Center erfolgt und von jeweils einer der Gruppenleiter interdisziplinär unterstützt wird. Bei der anspruchsvollen Betreuungsaufgabe im Spagat zwischen den Disziplinen unterstützen sich die beiden Nachwuchsgruppenleiter gegenseitig und werden dabei von der Mentorin unterstützt. Zweimal im Jahr findet ein Reflexionstreffen der Gruppenleiter und der Mentorin statt mit Fokus auf die Rolle als Leiter/in und den allgemeinen Fortschritt der Forschungsgruppe. 

Ebenso halbjährlich findet ein Treffen aller Mitarbeiter*innen der Nachwuchsforschungsgruppe statt, in der jede*r den Forschritt des Promotionsprojektes vorstellt. So werden nächste Schritte interdisziplinär geprüft und Probleme können frühzeitig erkannt und diskutiert werden. Die Mentorin der Arbeitsgruppe nimmt an diesem Kolloquien teil. Anschließend an den Fortschrittsbericht der einzelnen Forschungsfelder, werden die interdisziplinären Arbeitspakete besprochen. Jedes der interdisziplinären Arbeitspakete wird von einer der beiden Gruppenleiter organisiert mit Unterstützung einer der Doktoranden. Dadurch sammeln die Doktoranden auch im Bereich der interdisziplinären Arbeits- und Projektplanung Erfahrung. Um ihnen außerdem zu ermöglichen, in die lehrende Rolle hineinzuwachsen, ist die Zweitbetreuung von Masterarbeiten durch die Doktoranden geplant. Zusätzliche Weiterqualifizierung der Promovierenden wird durch mindestes einer Teilnahme pro Jahr an Summer Schools und Kursen erreicht. Die Themenfelder der Weiterbildung beinhalten sowohl Methoden-Kompetenz als auch wissenschaftliches Schreiben und transdisziplinäres Arbeiten. Ein mehrmonatiger Forschungsaufenthalt entweder beim Verbundkoordinator oder in einer noch zu definierenden Institution wird angestrebt.

% FEHLT NOCH: Diskussionsrunden führender Personen aus der SOEF-Fachebene verfolgen und beiwohnen

Die akademische Qualifizierung der Nachwuchswissenschaftler*innen wird außerdem dadurch verstärkt, dass von Anfang an eine Kultur des Publizierens gepflegt wird und die Promotionen kumulativ sein werden. Dadurch werden die Doktoranden von Beginn an an peer-reviewte Publikationen herangeführt. Dies kann zuerst in disziplinäre und dann in interdisziplinären Publikationen münden. Da interdisziplinäres Veröffentlichen eine doppelte Herausforderung ist werden die Promovierenden schrittweise herangeführt.

\section{Institutionen, Zukunftsperspektiven und Ergebnisverwertung}
\textit{Darstellung und Motivation der beteiligten Institutionen sowie Zukunftsperspektiven für die jeweiligen Mitglieder der Nachwuchsgruppe (nicht grundfinanzierte außeruniversitäre Forschungsinstitute haben zusätzlich darzustellen, inwieweit den betreffenden Mitgliedern zeitlich befristete Freiräume eingerichtet werden können, sich zeitweise voll auf ihre Qualifikation zu konzentrieren), erwartetes Ergebnis, Anwendungspotenzial und angestrebte Ergebnisverwertung. Der Verwertungsplan muss konkrete Maßnahmen der Öffentlichkeitsarbeit und des Wissenstransfers (auch von Zwischenergebnissen) beinhalten}

% EUF
An der Europa-Universität Flensburg wurde mit der Einrichtung des Interdisziplinären Instituts für Umwelt-, Sozial- und Humanwissenschaften eine zentrale institutionelle Struktur für disziplin-übergreifende Fragestellungen geschaffen. Die Verankerung der sozial-ökologischen Forschung in Form einer eigenständigen Nachwuchsgruppe würde diese mit dem konkreten Forschungsthema Energiesuffizienz mit Leben füllen. Die Zusammenarbeit der wirtschaftsingenieur-lastigen Abteilung Energie- und Umweltmanagement sowie des sozialwissenschaftlich geprägten Norbert Elias Center gilt als große Bereicherung und fügt sich sowohl in die Forschungsarbeit des Interdisziplinären Instituts für Umwelt-, Sozial- und Humanwissenschaften, als auch in das Leitbild der Europa-Universität Flensburg ausgezeichnet ein. Die Stärkung der inter- und transdisziplinären Forschung und insbesondere des Themenschwerpunktes Energiewende und Gesellschaft entspricht der Identität und der zukünftigen Entwicklung der Hochschule.

\textit{NEC: bitte kurze Beschreibung NEC (nur wenige Säzte) - Kurzfassung von Bernds Vorschlag}

\textit{Wuppertal: bitte kurze Beschreibung WI (nur ein kurzer Abschnitt)}

% Zukunftsperspektiven für die jeweiligen Mitglieder
Nach Ablauf des Projektes werden die Nachwuchswissenschaftler*innen mit ihrer inter- und transdisziplinären Erfahrung in einem gesellschaftlich hochaktuellen Thema bestens aufgestellt sein um ihre wissenschaftliche Karriere in diesem Bereich weiterzuführen. An der EUF werden durch eine perspektivische Verstetigung des Forschungsfeldes Gesellschaftliche Transformation und Energiewende Optionen für eine die interdisziplinäre akademische Laufbahn in diesem Bereich geschaffen. In der letzen Phase der Nachwuchsforschungsgruppe wird der Zukunftsplanung und Analyse der Optionen und Folge-Projekte der beteiligten Wissenschaftler*innen Raum gegeben. 

% Wuppertal Institut?

% Welches Ergebnis erwarten wir und wie kann es von Forschung und Zivil-Gesellschaft genutzt werden?
In Abschnitt \ref{sec:ziel} werden die Ziele der Nachwuchsforschungsgruppe aufgeführt. Der erarbeitete Methode zur konsistenten Szenarienerstellung kann von anderen, speziell interdisziplinär arbeitenden Forschern angewendet und weiterentwickelt werden. Auch das Energiesystem-Modell, dessen Programmier-Code und Daten komplett zur Verfügung gestellt wird bildet eine Basis um auf den Erkenntnissen zu Energie-Suffizienz aufzubauen. Dies ist vor allem durch die Offenlegung des gesamten Modellierungszyklunf von Original-Daten über Szenarien bis hin zu Ergebnis-Verifizierung möglich. Die entwickelten Szenarien selbst erweitern die Landkarte  möglichen Energiewende-Pfade durch das Aufzeigen von Suffizienz-Potentialen sowie Maßnahmen. Erarbeitete Handlungsoptionen für Suffizienz-Politik unterstützen Entscheidungsträgern von kommunaler bis EU-Ebene einen Beitrag zur nachhaltigen Transformation zu leisten.

% Öffentlichkeitsarbeit und Wissenstransfer (konkrete Maßnahmen)
Der wissenschaftliche Wissenstransfer wird durch die Teilnahme an Konferenzen und Workshops sowie dem Forschungsnetzwerk Energiewende sichergestellt. 
% konkrete Netzwerkde von NEC oder Wuppertal?
Ein Lernziel der Nachwuchsforschung ist nicht nur die wissenschaftlichem Verschriftlichung aber auch die Kommunikation der Forschungsergebnisse an Nicht-Wissenschaftler. Der Wissenstransfer soll in Form eines Blogs zum Thema Energie-Suffizienz im weiteren Sinne, einer Ringvorlesung sowie der Teilnahme der Wissenschaftler*innen an Formaten wie ``Science Slam'' geschehen. Die Workshops mit den Praxispartnern sind auch Teil der Öffentlichkeitsarbeit und dienen dem gegenseitigen Wissenstransfer von Zivilgesellschaft, Politik und Wissenschaft. Eine gemeinsame Exkursionsreihe 'Suffizienz-Beispiele in der Praxis' in Flensburg in Zusammenarbeit mit den Praxispartnern wäre wünschenswert.

\section{Zeitplanung und Kostenschätzung}
\textit{Gesamtkosten bzw. -ausgaben, Grobkalkulation von Personal-, Sach- und Reisemitteln, gegebenenfalls Berücksichtigung von Eigenbeteiligung sowie Drittmitteln).}

\textit{Basierend auf den geltenden tarifrechtlichen Regelungen und projektbezogen, können in der Regel maximal vier wissenschaftliche Personalstellen (teilbar) je Nachwuchsgruppe beantragt werden (davon maximal zwei Post-Doktorandinnen oder Post-Doktoranden). Bereits durch öffentliche Mittel grundfinanzierte Stellen können grundsätzlich nicht gefördert werden.
In begrenztem Umfang können auch Assistenz- und Hilfskräfte sowie Sachmittel und Mittel zur Einbindung von Praxispartnern beantragt werden.}
% Brauchen wir ein Gantt-Chart?
Die zeitliche Einordnung und Abfolge der Arbeitspakete kann Abbildung \ref{fig:forschungsprogramm} entnommen werden. Die Kalkulation der Personal-, Reise-, Sach- und Weiterbildungsmittel sind Tabelle \ref{tab:kostenkalkulation} zu entnehmen.

Für die gesamten fünf Jahre werden durchgängig vier Stellen beantragt von denen 1,75 Stellen am Wuppertal Institut und 2,25 Stellen an der Europa-Universität Flensburg verortet sind. Letztere setzen sich aus einer 0,75 die Leitungsstelle (TVL14) und drei Doktoranden-Stellen zusammen (halbe Stellen). Am Wuppertal Institut wird ebenso eine 0,75 Leitungsstelle (TVL14), sowie eine halbe Stelle als Doktorandenstelle (TVL13) und eine halbe Stelle für Betreuungs- und Mentorenarbeit (TVL14) eingeplant. Zusätzlich werden wissenschaftliche Hilfskräfte für 20 Stunden pro Woche und Forschungsfeld eingeplant. Auf die Personalmittel wird ein Overhead von 20\% für die EUF und von 90\% für die außeruniversitäre Forschungseinrichtung Wuppertal Institut veranschlagt.
% Wie erklären wir die weiteren 0,5 Stelle

In den Reisemitteln sind die Teilnahme an Konferenzen ab dem zweiten Jahr sowie an vier Workshops pro Person und Jahr enthalten. Außerdem wurde mit halbjährlichen Treffen in abwechselnd Wuppertal und Flensburg mit allen Personen und zusätzlichen drei bilateralen Treffen pro Jahr kalkuliert. In den Sachmitteln sind neben Literatur und Computer für die Modellierung, ab dem zweiten Jahr Gelder für Open Access Veröffentlichungen enthalten. In die Kategorie Weiterbildung fallen Kursgebühren sowie Summer Schools.

Tabelle \ref{tab:kostenkalkulation} fasst die Kosten je Insititution sowie die beantragte Zuwendung entsprechend der Förderquote zusammen.

\begin{table}[h]
\begin{center}
  \caption{Abschätzung der Gesamtkosten je Kostenkategorie für Energie- und Umweltmanagement (EUM) und Norbert Elias Center (NEC) (beide interdisziplinäres Institut der Europa-Universität Flensburg) und Wuppertal Institut}
  
\begin{tabular}[h]{|l | r | r | r | r|}
\hline
&&&&\\
& EUF EUM & EUF NEC & Wuppertal Institut & \textbf{Summe in Euro}\\
\hline
\hline
&&&&\\
 Personalmittel & & & & \\
 \hline
 &&&&\\
 Reisemittel & & & & \\
 \hline
 &&&&\\
 Sachmittel & & & & \\
 \hline
 &&&&\\
 Weiterbildung & & & &\\
 \hline
 \hline
 &&&&\\
 \textbf{Summe}&& \textbf{}&\textbf{}&\underline{\textbf{}}\\
 \hline
 \end{tabular}
 \label{tab:kostenkalkulation}
\end{center}
\end{table}

\begin{table}[h]
\begin{center}
  \caption{Abschätzung von Gesamtkosten und beantragter Förderung je Institution}
\begin{tabular}[h]{|l | r | r | r|}
\hline
&&&\\
Institution & Kosten [Euro] & Förderquote [\%] & \textbf{beantragte Zuwendung}\\
\hline
\hline
 &&&\\
 EUF & & 100 &\\
 \hline
 &&&\\
 Wuppertal Institut & & 90 &\\
 \hline
 \hline
 &&&\\
 \textbf{Gesamt} & & &\underline{\textbf{}}\\
 \hline
 \end{tabular}
 \label{tab:kostenkalkulation2}
\end{center}
\end{table}

\clearpage

%\sectionp*{Literatur} \label{sec:lit}
%\bibliographystyle{plainnat}%elsarticle-num}
\bibliographystyle{elsarticle-num}
%\bibliographystyle{apalike}
\bibliography{literature.bib}

\clearpage
\appendix

\section{Anhang}

Literaturlisten, Lebensläufe und gegebenenfalls Interessensbekundungen von Praxispartnern sind im Anhang beizufügen.

\end{document}
