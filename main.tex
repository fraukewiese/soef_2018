
\documentclass[a4paper,11pt,twoside]{scrartcl}
\usepackage[utf8x]{inputenc}
\usepackage{graphicx}
\usepackage{geometry}
\usepackage[ngerman]{babel}
%\usepackage{babelbib}
%\usepackage[backend=biber]{biblatex}
\usepackage{units}
\usepackage{url}
\usepackage{setspace}
\usepackage[
	pdftitle={Energie_Suffizienz},
 	pdfauthor={et.al.},
% 	pdfsubject={},
% 	pdfkeywords={},
	pdfstartview=FitH, % Auf Seitenbreite anpassen (Anzeige)
	pdfborder={0 0 0},
%  	bookmarks=true,
% %	plainpages=false,
	colorlinks=false,
	hyperfootnotes=false,
	pagebackref=false]{}
\usepackage[colorinlistoftodos,prependcaption,textsize=normalsize]{todonotes}  % disable
\usepackage{amsmath}
\usepackage{amstext}
\usepackage{amssymb}
\usepackage{color}
\usepackage[numbers]{natbib}
\usepackage{pdfpages}
\usepackage{hyphenat}
\usepackage{pdflscape} % für drucken ändern auf package lscape /pdflscape
\usepackage{textpos}
\usepackage{microtype}
\usepackage{enumitem}
\usepackage{multirow}
\usepackage{pdfcomment}

\usepackage{pgfgantt} % gantchart für den Arbeitsplan

% \usepackage{hyphenat}
\usepackage{float}
\usepackage{placeins}
\RequirePackage[bf]{caption}

\renewcommand{\textfraction}{.01} % vorher: .2
\renewcommand{\floatpagefraction}{.99}% vorher: .5
\renewcommand{\topfraction}{0.9}	% max fraction of floats at top
\renewcommand{\bottomfraction}{0.9}	% max fraction of floats at bottom

\newcommand{\ltab}{\raggedleft\arraybackslash}
\newcommand{\ctab}{\centering\arraybackslash} 
\newcommand{\rtab}{\raggedright\arraybackslash}

\usepackage{tabularx}
\newcolumntype{L}[1]{>{\raggedright\arraybackslash}p{#1}} % linksbündig mit Breitenangabe
\newcolumntype{C}[1]{>{\centering\arraybackslash}p{#1}} % zentriert mit Breitenangabe
\newcolumntype{R}[1]{>{\raggedleft\arraybackslash}p{#1}} % rechtsbündig mit Breitenangabe

\newcommand{\rem}[1]{}

% \renewcommand{\figurename}{Abb.}
% \renewcommand{\tablename}{Tab.}

\usepackage{acronym}
%\renewcommand{\bflabel}[1]{\normalfont{\normalsize{#1}}\hfill}

\usepackage[automark]{scrpage2}
\pagestyle{scrheadings}
\clearscrheadfoot
\ohead{\headmark}
\ofoot{\pagemark}
\setheadsepline{0.4pt}
\setfootsepline{0.4pt}

\setkomafont{pageheadfoot}{\rmfamily\small}

\renewcommand{\thefigure}{\arabic{section}-\arabic{figure}}
\renewcommand{\thetable}{\arabic{section}-\arabic{table}}

\newcommand{\entspricht}{\mathrel{\widehat{=}}}

\geometry{left=25mm,right=25mm, top=28mm, bottom=28mm}
\parindent 0pt
\parskip 11pt

% kein Platz zwischen \items
\setlist{nosep}
% Platz nach Überschriften reduzieren
\usepackage{titlesec}
\titlespacing*{\section}{0pt}{0pt}{0pt}
\titlespacing*{\subsection}{0pt}{0pt}{0pt}
\titlespacing*{\subsubsection}{0pt}{0pt}{0pt}
\titlespacing*{\paragraph}{0pt}{0pt}{5pt} % horizontal spacing in paragraph

\begin{document}
\onehalfspacing

\clearpage


{\singlespacing

\thispagestyle{empty}
\begin{center}

%first row logos
\begin{figure}[htb]
    \centering
    \begin{minipage}[c]{0.3\linewidth}
        \centering
        \includegraphics[width=5cm]{logos/WI_Logo_CMYK.pdf}
    \end{minipage}
    \hfill
    \begin{minipage}[c]{0.35\linewidth}
         %\centering
         \includegraphics[width=6cm]{logos/AbtEUM.JPG}
     \end{minipage}
    \hfill
    \begin{minipage}[l]{0.25\linewidth}
        %\centering
        \includegraphics[width=4.5cm]{logos/NEC_logo.jpg}
    \end{minipage}
\end{figure}

\iffalse
%second row logos
\begin{figure}[htb]
    \centering
    \begin{minipage}[c]{0.3\linewidth}
        \centering
        \includegraphics[width=2.2cm]{logos/2015_Logo_TUM_RGB.jpg}
    \end{minipage}
    \hfill
    \begin{minipage}[c]{0.35\linewidth}
        \centering
        \includegraphics[width=5.5cm]{logos/isea_rwth_logo.jpg}
    \end{minipage}
    \hfill
    \begin{minipage}[c]{0.3\linewidth}
        \centering
        \includegraphics[width=3cm]{logos/TU_Logo_lang_RGB_rot.png}
    \end{minipage}
\end{figure}
\fi
\vspace*{1.5 cm}

{\LARGE\textbf{\textsf{Skizze Nachwuchsforschungsgruppe}}

\textsf{\textit{zur Bekanntmachung \glqq inter- und transdisziplinär arbeitende\\ Nachwuchsgruppen im Rahmen der Sozial-ökologischen Forschung\grqq} }
}

\vspace{0.5cm}

{\Huge
\textbf{\textsf{Die Rolle von Energie-Suffizienz in Energiewende und Gesellschaft\\
%oder\\
%Die Rolle von Energie-Suffizienz bei der Transformation zu einem nachhaltigen Energiesystem (RESTNE)\\
%oder\\
%Die Rolle von Energie-Suffizienz in der Energiewende
}}

\textbf{\textsf{}}
}

{\LARGE
\textbf{\textsf{Akronym:{EnSu}}}
}

\vspace{1.5cm}

\textit{Eingereicht durch die Verbundkoordinatorin}

{\parskip 0pt
Frauke Wiese\\
i$^{2}$ Interdisziplinäres Institut für Umwelt-, Human- und Sozialwissenschaften\\
Europa-Universität Flensburg, Auf dem Campus 1, 24943 Flensburg}

\vspace{0.3cm}

\textit{und den Verbundpartner}

{\parskip 0pt
Benjamin Best\\
Wuppertal Institut für Klima, Umwelt, Energie gGmbH\\
Döppersberg 19, 42103 Wuppertal Wuppertal}

\vspace{0.5cm}

An den Projektträger im DLR \\
AE 41 Globaler Wandel/Klima- und Umweltschutz, Sozial-ökologische Forschung \\
Heinrich-Konen-Straße 1, 53227 Bonn


\end{center}


\clearpage
}

\setcounter{page}{1}

\section{Zielstellung und gesellschaftlicher Bedarf}
\textit{Beschreibung der Problem- und Zielstellung sowie des gesellschaftlichen Bedarfs}

Eine der zentralen gesellschaftlichen Herausforderungen ist die Transformation des Energiesystems. Wie kann die Versorgung mit Strom, Wärme, Mobilität sichergestellt werden während Klimaziele erreicht, soziale Gerechtigkeit gewahrt und die natürlichen Grenzen unseres Planeten auch langfristig eingehalten werden? 

Energiesystem-Modelle haben sich als Werkzeuge etabliert, um technisch mögliche und ökonomisch vorteilhafte Energiewende-Pfade zu analysieren. Sie helfen dabei, komplexe Zusammenhänge zwischen technischen Möglichkeiten (z.B. Flexibilität), Umweltbedingungen (z.B. Wetter-abhängige Erneuerbare), Marktregeln (z.B. Energy-only-Markt) im Licht klimapolitischer Zielsetzungen zu verstehen und unterstützen somit die Klima- und Energiepolitik.

Ein entscheidender Einflussparameter für berechnete Szenarien ist die zukünftige Nachfrage nach Strom, Wärme und Mobilität. Die große Überzahl der Modelle legt den Fokus jedoch auf die Bereitstellung der Energiedienstleistungen. Die Nachfrage-Seite wird meist nur innerhalb enger Grenzen variiert. Zwar wird die Verschiebung von z.B. Brennstoff- zu Stromnachfrage durch Elektrifizierung im Wärme- und Transportsektor betrachtet, jedoch kaum die absolute Nachfrage nach Wärme, Strom und Mobilität in Frage gestellt.

Nicht nur in der Modellierung von Energiewende-Pfaden, sondern auch generell bei Klimaschutz-Strategien liegt der Fokus bisher auf technischen Lösungen auf Seiten der Generierung von Strom, Wärme und Mobilität \cite{Creutzig2018}. 

Zwar bieten laut Modellrechnungen Erneuerbare Energien und Effizienz die Möglichkeit im Jahr 2050 80-95 Prozent Treibhausgas-Reduktion in Deutschland \cite{BMWi2017} zu erreichen. Es gibt jedoch verschiedene Gründe keine der drei Nachhaltigkeitsstrategien Konsistenz,(Erneuerbare Energien ersetzen Fossile), Effizienz (relative Reduktion des Energieverbrauchs bei Bereitstellung der gleichen Energiedienstleistung) und Suffizienz (absolute Reduktion der Nachfrage nach Energiedienstleistungen durch veränderte soziale Praktiken ohne Einbuße des menschlichen Wohlbefindens) außer Acht zu lassen.

Zum einen setzten die ambitionierten Ziele des Pariser Klimaabkommens die Herausforderung nochmal auf eine neue Stufe \cite{Rogelj2018}. Außerdem sprechen Askpekte wie Flächenverbrauch (Referenz fehlt noch), Ressourcenbedarf (Referenz fehlt noch) und Akzeptanzfragen \cite{Fuchs2016}, zusammengefasst die Einhaltung der planetaren Grenzen \cite{Rockström2009} dafür, die Chancen die Suffizienz bietet nicht zu übersehen.

Desweiteren ist bisher nicht abschließend geklärt, in welchem Ausmaß bisherige Emissionsreduktionen in Deutschland durch Konsistenz und Effizienz tatsächlich erreicht wurden. Verlagerungseffekte in andere Länder könnten ebenso eine Rolle gespielt haben \cite{Wiedmann2015}, so dass die Emissionsreduktionen bei Bilanzierung nach Verursacherprinzip geringer ausfallen. Auch die Rolle, die Effizienz spielen kann ist nicht klar. Zielszenarien für Deutschland gehen von einer Reduzierung der Stromnachfrage durch eine massive Steigerung von Effizienz aus \cite[Modul 1]{BMWi2017}. Zugleich findet jedoch in der fachwissenschaftlichen und energiepolitischen Diskussion der Befund Aufmerksamkeit, dass Energieeffizienzsteigerungen in der Regel von Rebound-Effekten begleitet werden, welche die Einsparungen teilweise kompensieren oder mitunter sogar zu einem verstärkten Energieverbrauch („Backfire“) führen \cite{DeutscherBundestag2013,Santarius2012}. Auch aufgrund solcher Rebound-Effekte wurden in den vergangenen Jahren die Energiesparpotenziale nicht voll ausgeschöpft, was dazu beitrug, dass in Deutschland trotz aller Maßnahmen zur Verbesserung der Energieeffizienz der Stromverbrauch im Jahr 2016 gegenüber 2008 (acht Jahre) um nur 1,5 Prozent gesunken ist \cite{UBA2017}. Dies lässt das erklärte Ziel weitere 8,5 Prozent bis 2020 (vier Jahre) zu schaffen sehr ambitioniert erscheinen.

In Anbetracht des möglichen Beitrags der Suffizienz und der Größe der Herausforderung, ist die Rolle von Energie-Suffizienz derzeit unterrepräsentiert in Diskussion und Forschung für Klimaschutz und Energiewende. Abbildung \ref{fig:zusammenspiel} verdeutlicht schematisch den Gedanken des Zusammenspiels der drei Dimensionen für die Erreichung der Klimaziele. 


\begin{figure}[!h]
    \centering
    \includegraphics[width=0.8\textwidth]{figures/Zusammenspiel2.pdf}
    \caption{Schamatisch: Bisherige Emissionsreduktion durch Konsistenz (links), notwendiger Beitrag von Effizienz und Suffizienz in Zukunft (rechts) zur Erreichung der Klimaziele in Deutschland}
    \label{fig:zusammenspiel}
\end{figure}

Weiter hat sich in der Forschung die Erkenntnis durchgesetzt, dass nur unter Einbeziehung und dem Zusammenspiel verschiedener Disziplinen - wie technisch-/ingenieurwissenschaftlicher Fachrichtungen, Politik-, Wirtschafts- und Sozialwissenschaften - die gesellschaftliche Herausforderung der Transformation des Energiesystems erfolgreich bewältigen lässt \cite{WBGU2011}.

Besonders deutlich wird die Notwendigkeit eines Zusammenspiels der Disziplinen bei der Erstellung von Energie-Szenarien. Gesellschaftliche Entwicklungen beeinflussen sowohl die Nachfrage nach Energie-Dienstleistungen als auch politische Rahmenbedingungen sowie die technisch-ökonomischen Optionen zur Erfüllung der Energienachfrage (schematisch dargestellt in \ref{fig:szenarien}). Gängige Praxis in der Energiesystem-Modellierung ist es, sich auf letztere zu beschränken, die gesellschaftliche Entwicklung ist ein eigenes Forschungsfeld (\ref{fig:szenarien} links). Es ist jedoch eine Voraussetzung für konsistente Energie-Szenarien, diese in gesellschaftliche Zukünfte einzubetten und damit gesellschaftliche Entwicklung,technische Optionen und Politik stärker zusammenzudenken (\ref{fig:szenarien} rechts). Außerdem ermöglicht die Erweiterung des Bezugsrahmen um die gesellschaftliche Perspektive, Suffizienz neben Effizienz und Konsistenz in der Modellierung abzubilden. 

\begin{figure}[!h]
    \centering
    \includegraphics[width=0.7\textwidth]{figures/Szenarien.pdf}
    \caption{Rahmen für Energie-Szenarien: Links: Getrennte Betrachtung / Rechts: Einbettung der Annahmen für Energie-Nachfrage, Politikrahmen und technische Optionen in gesellschhaftliche Entwicklungen}
    \label{fig:szenarien}
\end{figure}

Ziele der Nachwuchsforschungs-Gruppe Energie-Suffizienz sind:
\begin{itemize}
 \item Einen bisher unterrepräsentierten Teil der Energiewende -- Suffizienz -- erforschen
  \item Eine interdisziplinäre Methode für die Erstellung von konsistenten Energieszenarien (Bezugsrahmen für diese konsistenten Szenarien siehe Abbildung \ref{fig:szenarien}.
  \item (Weiter)-Entwicklung eines Energiesystem-Modells, in dem Konsistenz, Effizienz und Suffizienz integriert betrachtet werden können
 \item Einen möglichen und eventuell notwendigen Beitrag von Energie-Suffizienz zur Erreichung der Klimaziele auf nationaler (Deutschland) und kommunaler (Flensburg) Ebene ermitteln
 \item Zukunftserzählungen von gesellschaftlichen Suffizienz-Pfaden 
 \item Handlungsoptionen für Suffizienz-Politik von kommunaler bis EU-Ebene
 \item Literatur und Methoden der Energiesystem-Analyse als auch die der sozial-ökologischen Transformationsforschung erweitern. Schwerpunkt liegt dabei auf Synergien zwischen den Forschungsgebieten.
 \item Das Methodenspektrum junger Forscher mit Hintergründen aus Sozialwissenschaft (sozial-ökologische Transformationsforschung) Wirtschaftingenieurwesen (Energiesystem-Analyse) und Politikwissenschaft interdisziplinär erweitern
\end{itemize}

Daten und Software die im Rahmen des Projektes erarbeitet werden, werden open source zur Verfügung gestellt, was die Verwendbarkeit der Forschungsarbeit für andere ermöglicht. Der Open Science Ansatz der der Nachwuchs-Forschunggruppe zugrunde liegen wird, beinhaltet auch den offenen Zugang zu Publikationen etc. Die Erkenntnisse sollen außerdem politische Entscheidungsprozesse zu Klimaschutz und Energiewende unterstützen und Empfehlungen für Suffizienzpolitik beinhalten. Ein Schwerpunkt wird die Ergebnis-Kommunikation auch im populär-wissenschaftlichen Bereich, um auch eine außerwissenschaftliche Öffentlichkeit zu erreichen.

\section{Stand von Wissenschaft und Technik sowie eigene Vorarbeiten}
\subsection*{Energiesystem-Analyse und Modellierung}
\begin{itemize}
    \item Energiesystem-Analyse und Modellierung: wichtige Rolle auch in policy advice, hochkomplexe Modelle, inzwischen auch Open Source Modelle, wenn auch noch Mangel an Transparenz
    \item Generelle Herausforderungen ESM:
    \begin{itemize}
     \item Open heißt nicht unbedingt verständlich, bisher Mangel an Ergebniskommunikation 
     \item Sozialwissenschaftlicher Komponenten bisher unterrepräsentiert wenn es auch auf dem Gebiet Akzeptanz
     \item Arbeiten im Bereich Energie-Nachfrage beziehen sich viel auf Flexibilität der Nachfrage, weniger auf die absolute Nachfrage nach Strom, Wärme, Mobilität
    \end{itemize}
    \item Eigene Vorarbeiten (Frauke): ESM, open ESM, kritische Betrachtung ESM, Verbindung über techn.-ökono. Sichtweise hinaus, bereits interdisziplinär gearbeitet / (EUM-Team): oemof, open, Klimaschutz (transdisziplinär)
\end{itemize}

\subsection*{Sozial-ökologische Transformationsforschung}
% NEC: bitte ergänzen, v.a. in Bezug auf Energiewende und Suffizienz inklusive eigene Vorarbeiten
% Rebound text von Bernd, siehe auch Bezug zu Nachwuchsgruppe Nachwuchsforschungsgruppe Rebound, Suffizienz und Digitalisierung)
    %https://www.fona.de/de/nachwuchsfoerderung-sozial-oekologische-forschung-20620.html
    %  1.05.2016 - 30.04.2021
    % Digitalisierung und sozial-ökologische Transformation. Rebound-Risiken und Suffizienz-Chancen digitaler Dienstleistungen
    
\subsection*{Sufizienz-Politiken}
% Wuppertal: bitte ergänzen, v.a. in Bezug auf Energie-Suffizienz inklusive eigene Vorarbeiten


\section{Bezug zur Sozial-ökologischen Forschung und zu den Förderzielen}

\url{https://www.fona.de/mediathek/pdf/SOEF_Foerderkonzept_barrierefrei.pdf}

SEHR GERN ERGÄNZEN

%Ideen:
%interdisziplinäres Verständnis wird geschaffen und baut auf dem disziplinären Vorwissen auf indem teilweise gleiche Arbeitspaket erst disziplinär parallel ausgeführt werden um dann in der Synthesephase den gemeinsamen Blick drauf zu werden (z.B. Szenarien-Methoden)
%...gemeinsamen Methodenentwicklung vor dem Hintergrund transdisziplinärer Nachhaltigkeitsforschung....

% aus altem Antrag
Die Weiterentwicklung institutioneller Kapazitäten zur Durchführung transdisziplinärer Nachhaltigkeitsforschung wird durch die Verankerung der Nachwuchsgruppe am Interdisziplinären Institut für Umwelt-, Sozial und Humanwissenschaften an der Europa-Universität Flensburg geschaffen. In der eigenverantwortlichen Arbeitsgruppe bekommen junge Wissenschaftler*innen die Möglichkeit sich auf der Basis ihres disziplinären Vorwissens sozial-ökologischen Fragestellungen zu  widmen. Möglichkeiten zur Weiterqualifizierung werden durch die  Nachwuchsgruppe für wissenschaftliche Mitarbeiter*innen geschaffen, die zwar schon jetzt an den Schnittstellen forschen, dafür aber nicht den entsprechenden Rahmen haben um sich auch wissenschaftlich zu qualifizieren.


\section{Forschungsarbeit, Arbeitsprogramm, Methoden, Disziplinen}
\textit{Beschreibung der geplanten Forschungsarbeiten und des Arbeitsprogramms, unter Einschluss der Darstellung von Methoden, die zur Anwendung kommen bzw. entwickelt werden sollen; sowie der disziplinären Zusammensetzung der geplanten Nachwuchsgruppe}

%%%%%%%%%%%

\begin{figure}[!h]
    \centering
    \includegraphics[width=0.8\textwidth]{figures/Forschungsarbeit.pdf}
    \caption{Forschungsprogramm: Disziplinäre Forschungsarbeit (vertikal) und  interdisziplinäre (horizontal) Arbeitspakete im Verlauf der fünf Jahre sowie geplanter disziplinärer und interdisziplinärer Output}
    \label{fig:forschungsprogramm}
\end{figure}

Abbildung \ref{fig:forschungsprogramm} gibt einen Überblick wie die disziplinären Forschungsfelder (vertikal dargestellt) im Verlauf der fünf Jahre über die interdisziplinären Arbeitspakete (horizontal dargestellt) ineinander greifen. Im Folgenden wird jedes Forschungsfeld mit Meilensteinen sowie die interdisziplinären Arbeitspakete kurz erläutert.

\subsection*{Forschungsfeld sozial-ökologische Transformationsforschung: Gesellschaftlicher Wandel und Energieverbrauch}
% NEC: Bitte umformulieren/ergänzen
Der Zusammenhang zwischen gesellschaftlichem Wandel und Energieverbrauch soll zuerst historisch untersucht werden, um dann zukünftige Entwicklungen zu entwerfen. Mit Methoden des Transformationsdesigns wird dann ...

\textbf{Meilensteine}
\hline
T1: Zusammenhang gesellschaftlicher Wandel und Energieverbrauch - Vergangenheit und Gegenwart\\
T2: Zusammenhang gesellschaftlicher Wandel und Energieverbrauch - Zukunft\\
T3: Szenarien-Methode\\
...
T Output: Zukunftserzählungen von gesellschaftlichen Suffizienz-Pfaden
\hline

\subsection*{Forschungsfeld Politikwissenschaft: Suffizienz-Politiken und Rahmenbedingungen}
% Wuppertal: Bitte umformulieren/ergänzen
\textbf{Meilensteine}
\hline
Ö1: Regulatorische und politische Rahmenbedingungen für Konsistenz, Effizienz und Suffizienz - Vergangenheit und Gegenwart\\
Ö2: Regulatorische und politische Rahmenbedingungen für Konsistenz, Effizienz und Suffizienz - Zukunft\\
Ö3: Szenarien-Methode\\
Ö4: Quantifizierung der Auswirkungen (Emissionsreduktion, Kosten) verschiedener regulatorischer Maßnahmen zu Konsistenz, Effizienz und Suffizienz\\
Ö5: \\
Ö Output: Zusammenfassung Konsistenz, Effizienz und Suffizienz Maßnahmen auf individueller, kommunaler, nationaler und EU-Ebene
\hline

\subsection*{Forschungsfeld Systemanalyse: Integrierte Energie-System-Modellierung: Beitrag von Konsistenz, Effizienz und Suffizienz zur Energiewende}

\textbf{Meilensteine}
\hline
S1: Auswahl eines offenen Energiesystem-Modells (erweiterbar, anpassbar, open source, gewisser Nutzerkreis)\\
S2: Erweiterung ESM um die ermittelten Nachfrage-Parameter\\
S3: Szenarien-Methode\\
S4: Modellierung: Quantifizierung des Beitrags von Konsistenz, Effizienz und Suffizienz zu einem Energiesystem mit geringem Anteil fossiler Brennstoffen\\
S5: Ergebnis-Kommunikation\\
S Output: Integriertes ESM für Konsistenz, Effizienz, Suffizienz wird unter einer offenen Lizenz Wissenschaft und Gesellschaft zur Verfügung gestellt
\hline

\subsection*{AP1: Interdisziplinärer Forschungsrahmen}
Am Beginn der Forschungsgruppe steht der gemeinsame Entwurf eines interdisziplinären Bezugsrahmens für Forschung zu Energie-Suffizienz aus Sicht der Energiesystemanalyse, des Transformationsdesigns und aus ökonomischer/regulatorischer Sicht. Neben dem inhaltlichen "aufeinander einlassen" werden auch die Formen und Formate für die gemeinsame Zusammenarbeit entworfen, sowohl inhaltliche, also auch organisatorische wie z.B. Häufigkeit der Treffen etc.

\subsection*{AP2: Gesellschaftliche Indikatoren für Energieverbrauch}
% NEC liefert Textbaustein / passt an
Um Suffizienz im Energiesystem-Model abbilden zu können bedarf es einer Übersetzung von gesellschaftlichen Indikatoren zu quantifizierbarem Energieverbrauch. Dieser sozioligisch-technisch-ökonomischen Schnittstelle widmet sich dieses AP. Das Forschungsfeld Transformationsdesign ermittlet relevante gesellschaftliche Einflussfaktoren auf die Energienachfrage. Diese bilden die Grundlage für die Auswahl der Indikatoren, die in das Energiesystem-Modell als Input-Parameter aufgenommen werden. Ein Beispiel zur Verdeutlichung: Das gestiegene Durchschnittsalter der Bevölkerung führt zu gesteigerter Wohnfläche, was sich wiederum in Wärmebedarf ausdrücken lässt. Oder: Eine Reduktion der durchschnittlichen Arbeitszeit führt zu einem erhöhten Mobilitätsaufkommen (BESSERE BEISPIELE EINFÜGEN). Im Forschungsfeld Ökonomie wurden relevante regulatorische Einfussfaktoren auf Suffizienz, Konsistenz und Effizienz ermittelt sowie mögliche zukünftige Politik-Maßnahmen erdacht. In interdisziplinärer Zusammenarbeit wird die Übersetzung ins Energie-System-Modell methodisch erarbeitet. Diese bildet die Grundlage für die Erweiterung des Energiesystem-Modells.
%Das kann zum Beispiel die EnEV sein, die als Effizienz-Maßnahme zu verstehen ist, da sie nicht auf den Wärmebedarf pro Person sondern auf den Wärmebedarf pro Wohnfläche abzielt. 

\subsection*{AP3: Szenarien-Methode}
Dieses methodische Arbeitspaket wird aus disziplinärer Sicht vorbereitet: Verschiedene Methoden zur Szenarienerstellung werden auf ihre Eignung für interdisziplinäre Szenarien geprüft die alle Kompenenten aus Abbildung \ref{fig:szenarien} integriert betrachten. Ein Auswahlkriterium soll hierbei die Eignung für die Beteiligungung von Praxispartnern in der Szenarienerstellung sein. Im Anschluss an die disziplinäre Vorarbeit werden wird in interdisziplinärer Zusammenarbeit eine passende Methode zur Szenarien-Erstellung ausgewählt oder bestehende disziplinäre kombiniert und wo notwendig erweitert. MÖGLICHE METHODEN NENNEN (e.g. Cross-Impact-Balance .....)

\subsection*{AP4:Szenarien-Erstellung}
Im Anschluss an die Auswahl der Methode werden in dieser inter- und transdisziplinären Arbeitsphase konsistente Gesellschafts-Energiesystem-Szenarien erstellt. Im Rahmen von Workshops und Befragungen werden Praxipartner in die Szenarienerstellung einbezogen. SZENARIO-FOKUS FESTLEGEN: z.B. SZENARIEN FÜR FLENSBURG (kommunal) UND SZENARIEN FÜR DEUTSCHLAND?

\subsection*{AP5:Iteration Szenarien}
Die konkrete Übersetzung der Szenarien in quantifizierte Daten als Modell-Input macht mindestens eine Iterationsschleife und eine Anpassung oder Ergänzung der Szenarien erforderlich. Die Ergebnisse werden interdisziplinär verifiziert und validiert (METHODE NENNEN/REFERENZIEREN?) und im Anschluss wo notwendig berichtigt, erweitert und angepasst.

\subsection*{AP6:Synthese und Ausblick}
Im letzten Teil des Projektes steht die Synthese-Phase der Ergebnis, die als Ergänzung zu disziplinären Publikationen einen Fokus auf die die Publikation von interdisziplinären Methoden, Erfahung und praktische Verwertbarkeit der Ergebnisse legt. Dies geht Hand in Hand mit einer Ausblicks-Phase, in der Nachwuchs-Forscher*innen sich mit weiteren Qualifizierungs- und interdisziplinären Projektmöglichkeiten beschäftigen

\section{Kooperationen, Forschungs- und Praxispartner}
\textit{vorgesehene Kooperationen (Forschungs- und Praxispartner) und Arbeitsteilung, Einbindung der Praxispartner in den transdisziplinären Forschungsansatz}\\
Die Nachwuchsgruppe ist am interdisziplinären Institut für Umwelt-, Sozial- und Humanwissenschaften der Europa-Universität Flensburg (EUF) UND ...? angesiedelt. Das Forschungsfeld \textit{Gesellschaftlicher Wandel und Energieverbrauch} ist am Norbert Elias Center for Transformation Design & Research verortet. Die Abteilung Energie- und Umweltmanagement übernimmt den Teil der Energie-System-Analyse.
\begin{itemize}
 \item ÖKOMOMISCHER TEIL? KOOPERATION WUPPERTAL-INSTITUT?
 \item Kooperation Energie-System-Modellierung: open energy modelling initiative
 \item Praxispartner: Stadt Flensburg, Klimapakt Flensburg, ANDERE?
 \item Praxispartner vor allem für die Szenarien einbezogen, auch in der Erarbeitugn von Handlungsempfehlungen, gehen die Iterationsschleife Szenarien mit, so dass sie auch die Aufnahme ihreres Inputs wiederum bewerten können
\end{itemize}

\section{Betreuungskonzept}
\textit{(inklusive der/des vorgesehenen Mentorin bzw. Mentors und Nachweis deren/dessen Expertise in Bezug auf inter- und transdisziplinäre Forschung}

% NEC: 1-2 Sätze Betreuungskonzept am NEC

% Wuppertal: 1-2 Sätze Betreuungskonzept WI und Nachweis Expertise in Bezug auf inter- und transdisziplinäre Forschung der Mentorin.

\begin{itemize}
 \item disziplinäre Betreuung der Forschungsfelder in dem genannten Arbeitsgruppen (NAMEN NENNEN? Michaela Christ? ,...)
 \item interdisziplinären Arbeitspakte betreut Nachwuchsgruppenleiter
 \item evtl. zu jeder 75prozent doktorandenstelle noch eine disziplinäre 25prozent stelle von nem PostDoc, der dann auch noch mit Erfahrung weitergeben kann
 \item halbjährliche Workshops, bei denen sich alle Mitarbeiter*innen (auch Leitung) gegenseitig den Forschritt ihrer Arbeit, nächste Schritte, Probleme, etc. vorstellen und diskutieren. 1xjährlich Mentor mit dabei
 \item Summer schools, Auslands-Forschungsaufenthalt angestrebt, Methoden-Komptenz erweitern durch Teilnahme an Kursen, Weiterbildungen wissenschaftliches Arbeiten, wissenschaftliches Schreiben (mind. 1 Fortbildung pro Jahr + 1 online Kurs?)
 \item Unterstützung Gruppenleiter durch Mentor
 \item Doktoranden betreuen wiederum Master-Arbeiten um die interdisziplinäre Erfahrung weiterzugeben und auch durch die Betreuungserfahrung zu wachens
 \item evtl. auch Lehrerfahrung sammeln, soll aber sehr passend zum eigenen Thema sein
 \item kumulative Promotion, damit bereits zwischendrin Fokus auf wissenschaftliche Publikaitonen, Peer Review, Feedback gelegt wird
 \item wissenschaftliche Publikationen, erst disziplinär, dann auch interdisziplinär (oft schwieriger)
\end{itemize}

\section{Institutionen und Zukunftsperspektiven}
\textit{Darstellung und Motivation der beteiligten Institutionen sowie Zukunftsperspektiven für die jeweiligen Mitglieder der Nachwuchsgruppe (nicht grundfinanzierte außeruniversitäre Forschungsinstitute haben zusätzlich darzustellen, inwieweit den betreffenden Mitgliedern zeitlich befristete Freiräume eingerichtet werden können, sich zeitweise voll auf ihre Qualifikation zu konzentrieren), erwartetes Ergebnis, Anwendungspotenzial und angestrebte Ergebnisverwertung. Der Verwertungsplan muss konkrete Maßnahmen der Öffentlichkeitsarbeit und des Wissenstransfers (auch von Zwischenergebnissen) beinhalten}


An der Europa-Universität Flensburg wurde mit der Einrichtung des Interdisziplinären Instituts für Umwelt-, Sozial- und Humanwissenschaften eine zentrale institutionelle Struktur für disziplin-übergreifende Fragestellungen geschaffen. Die Wissenschaftler*innen verbindet unter anderem das Thema Nachhaltigkeit aus der jeweiligen disziplinären Sichtweise. Mit dem wöchentlich stattfindenden Interdisziplinären Kolloquium hat sich hier eine Austauschplattform für interdiszipinäre Fra-
gestellungen etabliert, die sich Themen wie Gesellschaft und Nachhaltigkeit (WiSe 2014/15) oder Raum und Gesellschaft (SoSe 2015) widmet (WELCHE AKTUELLEREN?). Die Verankerung der sozial-ökologischen Forschung in Form einer eigenständigen Nachwuchsgruppe an der Europa-Universität Flensburg gilt als große Bereicherung und fügt sich sowohl in die Forschungsarbeit des Interdisziplinären Instituts für Umwelt-, Sozial- und Humanwissenschaften, als auch in das Leitbild der Europa-Universität Flensburg ausgezeichnet ein. Die Stärkung der inter- und transdisziplinären Forschung entspricht der Identität und der zukünftigen Entwicklung der Hochschule.

% NEC: bitte kurze Beschreibung NEC (nur wenige Säzte) - Kurzfassung von Bernds Vorschlag


% Wuppertal: bitte kurze Beschreibung WI (nur ein kurzer Abschnitt)




\section{Zeitplanung und Kostenschätzung}
\textit{Gesamtkosten bzw. -ausgaben, Grobkalkulation von Personal-, Sach- und Reisemitteln, gegebenenfalls Berücksichtigung von Eigenbeteiligung sowie Drittmitteln).}

\textit{Basierend auf den geltenden tarifrechtlichen Regelungen und projektbezogen, können in der Regel maximal vier wissenschaftliche Personalstellen (teilbar) je Nachwuchsgruppe beantragt werden (davon maximal zwei Post-Doktorandinnen oder Post-Doktoranden). Bereits durch öffentliche Mittel grundfinanzierte Stellen können grundsätzlich nicht gefördert werden.
In begrenztem Umfang können auch Assistenz- und Hilfskräfte sowie Sachmittel und Mittel zur Einbindung von Praxispartnern beantragt werden.}
\begin{itemize}
    \item eine Leiterstelle (Postdoc) - möglicherweise teilen
    \item je einen Doktoranden je Strang, zu 75 prozent?
    \item auch Auslandsaufenthalte einplanen
    \item summer schools
    \item Fortbildungsmassnahmen Gruppenleiter
\end{itemize}

Die Zeitplanung findet sich in Abbildung \ref{fig:forschungsprogramm}. (TODO: gantt-chart?). Die Kostenschätzung pro Jahr ist der Tabelle \ref{tab:kostenkalkulation} zu entnehmen. Für die EUF sind drei Stellen inklusive der Leitungsstelle, für das ... ist eine Stelle kalkuliert. Die Zahlen für Personalmittel beinhalten einen Overhead von 20\%.

\begin{table}[h]
\begin{center}
  \caption{Kostenschätzung EUF}
\begin{tabular}[h]{lrrrrr}
& EUF EUM & EUF NEC & Wuppertal Institut & Summe in Euro\\
\hline
\hline
&&&&\\
 Personalmittel & & & & &\\
 \hline
 &&&&\\
 Reisemittel & & & & &\\
 \hline
 &&&&\\
 Sachmittel & & & & &\\
 \hline
 &&&&\\
 Weiterbildung & & & & &\\
 \hline
 \hline
 &&&&\\
 \textbf{Summe in Euro}&& \textbf{}&\textbf{}&\underline{\textbf{}}\\
 \label{tab:kostenkalkulation}
\end{tabular}
\end{center}
\end{table}

\clearpage

\appendix
\section{Anhang?}

Literaturlisten, Lebensläufe und gegebenenfalls Interessensbekundungen von Praxispartnern sind im Anhang beizufügen.

\clearpage
\section{Literatur} \label{sec:lit}
\bibliographystyle{plainnat}%elsarticle-num} 
\bibliography{literature.bib}

\end{document}
