
%IDEEN FORSCHUNGSSCHWERPUNKTE BEN FRAUKE

%UTAS VORSCHLAG
%Frauke: Schwerpunkt Interdisziplinäre Integration und Kopplung    ### hier müsstest Du einen Namen für das finden, was Du tust und worauf Du hinaus willst
%Ben: Schwerpunkt transdisziplinäre Integration und Kopplung.  ### dito. Vorschlag von mir: Akteurs- und Konstellationsanalyse des Energieregimes 

%MICHAELAS VORSCHLAG
%Andernfalls schlage ich vor, an ein, zwei Punkten im Text Variationen von „Die Erkenntnisse aus den Arbeitspaketen werden von den Forschungsgruppenleiter*innen einerseits in das entstehende Energiesystemmodell integriert und andererseits zur kritischen Reflexion bestehender energiepolitischer Steuerungsinstrumente genutzt.“

%Ich finde es echt schwer das für mich zu formulieren, ein paar Stichpunkte die mir bisher in den Sinn gekommen sind:
%Wissenstransfer zwischen qualitativer und quantitativer Energiewendeforschung
%Energiesystem-Modellierung in einen interdisziplinären Kontext setzen
%Analyse Nutzen der Energiewende-Modellierung (wo nützt sie und wo nicht)
%Erkenntnisgewinn aus der interdisziplinären Szenarienerstellung im Vergleich zu den disziplinären?
%Ergebniskommunikation von Energiesystemmodellen inklusiver ihrer Wirkweise: In welcher Form und zu was tragen sie zu Erkenntnisgewinn bei ... und wo auch nicht.
% mir geht es darum uns von den Modellen beim Rechnen helfen zu lassen, aber davon nicht diktieren zu lassen was der richtige Pfad ist. Szenarien lebensweltlich einbetten trifft es teilweise weil bei den Szenarien treffen sich sozusagen die echte mit der zahlen-modellierwelt. Allerdings muss dann auch die Ergebnisseite lebensweltlich eingebettet werden. Es sind nicht die Zahlen der Modellergebnisse sondern ihre Interpretation die weiterhelfen. Die Zahlen bringen aber manche Dinge so schoen auf den Punkt, dass die Modellierungsarbeit (incl. Szenarien und Ergebniskommunikation) Erkenntnisgewinn mit sich bringt.
% Niemand soll immer mehr haben wollen müssen. Wie fließt das ins Energiesystem-Modell ein. z.B. mehr shared economy, kürzere Arbeitszeit, weniger Produktion in der Industrie. Das hat eine geringere Energienachfrage zur Folge und so muss man weniger Windkraftanalgen aufstelellen und hat vll. nicht das Problem mit der knappen Biomasse-Resource und hat es leichter die Klimaziele zu erfüllen.

% Ben: Finde ich auch schwer. Hier meine Notizen. Das sind vor allem Stichworte, die mir bisher in der Skizze fehlen und durch mich (mehr oder weniger gut) abgedeckt werden könnten.
%- Suffizienzpolitik zur absoluten Minderung des Energieverbrauchs/ Energiesuffizienz als politische Praxis / Politisches Gestaltungsinstrumentarium für Suffizienzpolitik, insbesondere Wirtschaftspolitik und Raumordnung
%- Kognitive Wissensintegration - erfordert den Einsatz spezifischer Methoden der inter- und transdiszplinären Wissensintegration. 
%- Räumliche Wirkungen von Energieszenarien mit Suffizienz-Optionen


%Formatierung / Kürzung überlasse ich gern Dir, Frauke

Dr. des. Benjamin Best 

Benjamin Best ist seit 2011 wissenschaftlicher Mitarbeiter am Wuppertal Institut in der Abteilung Zukünftige Energie- und Mobilitätsstrukturen. Er arbeitet im interdispziplinären Forschungsprojekt DoNaPart, welches im Rahmen von BMBF-SÖF / Zukunftsstadt gefördert wird und Erkenntnisse gewinnen möchte, wie durch gemeinschaftliches Handeln nachhaltige und lebenswerte Städte geschaffen werden können. Best ist zudem seit 2014 am "Virtuellen Institut - Energiewende NRW" beteiligt, in dem nicht-technische Energieforschungsprojekte in NRW gebündelt werden. 
Best hat Soziologie, Geschichte (B.A.) und Sustainability Economics and Management (M.A.) in Oldenburg studiert. Er hat Ende April 2018 seine Doktorarbeit im Fach Politikwissenschaft an der Bergischen Univesität Wuppertal erfolgreich verteidigt. Best war 2014 Mitglied im Organisationskreis der „International Degrowth Conference“ in Leipzig und ist Juror des Karl W. Kapp-Forschungspreises für Ökologische Ökonomie. 
\fi