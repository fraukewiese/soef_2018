Die eingegangenen Projektskizzen werden unter Hinzuziehung von externen Sachverständigen nach folgenden Kriterien bewertet:

    Passfähigkeit zur Sozial-ökologischen Forschung und zur Bekanntmachung;
    wissenschaftliche Qualität und Originalität des Projektes;
    Kenntnis des Stands von nationaler wie internationaler Forschung und anderer einschlägiger Wissensquellen im Themenfeld;
    stringentes Forschungsdesign und angemessene Auswahl der Methoden bzw. Darlegung der zu entwickelnden Methoden;
    Kompetenz der Projektleitung und Qualität des Betreuungskonzepts;
    eine der Problemstellung angemessene interdisziplinäre Zusammensetzung des Forschungsteams (Beteiligung der für den gewählten Forschungsgegenstand relevanten Fächer und Kompetenzen);
    Relevanz und Rolle von Praxispartnern;
    anwendungsorientierte und wissenschaftliche Verwertungsperspektive (science and policy impact);
    Berücksichtigung der Genderperspektive (siehe Nummer 2.1).
    
    
   https://www.fona.de/de/nachwuchsfoerderung-sozial-oekologische-forschung-20620.html 
    
    Im Förderkonzept des Förderschwerpunktes „Sozial-ökologische Forschung“ sind neben der thematisch ausgerichteten Förderung (wie bspw. „Nachhaltiges Wirtschaften“ oder „Transformation Urbaner Räume“) auch strukturelle Förderformate vorgesehen, um die transdisziplinäre Nachhaltigkeitsforschung weiter zu stärken. Der Förderung des auf diesem Gebiet inter- und transdisziplinär arbeitenden wissenschaftlichen Nachwuchses kommt dabei eine besondere Bedeutung zu. Denn nach wie vor ist das Wissenschaftssystem vorwiegend disziplinär ausgerichtet. Interdisziplinäre und insbesondere transdisziplinäre Herangehensweisen werden nicht ausreichend honoriert. Insbesondere wissenschaftliche Karrieren sind so nur schwer aufzubauen, auch wenn die bisherige Nachwuchsförderung in der Sozial-ökologischen Forschung bereits konkrete Fortschritte gezeigt hat. Dass die Förderung in die richtige Richtung geht, aber bei weitem noch nicht ausreichend ist, zeigt die in 2012/2013 durchgeführte Evaluierung.

Ziel der Fördermaßnahme „Nachwuchsgruppen in der Sozial-ökologischen Forschung“ ist es, sowohl institutionelle als auch personelle Kapazitäten, die für die Durchführung inter- und transdisziplinärer Nachhaltigkeitsforschung benötigt werden, weiterzuentwickeln. Nachwuchswissenschaftlerinnen und Nachwuchswissenschaftler sollen mit fachübergreifenden Forschungsperspektiven an den Schnittstellen von Natur-, Ingenieur- und Gesellschaftswissenschaften die Gelegenheit erhalten, sich akademisch weiter zu qualifizieren und allgemein ihre Chancen für Karrierewege in inter- und transdisziplinärer Wissenschaft, Wirtschaft und Zivilgesellschaft zu verbessern. Bei der Bearbeitung einer selbst gewählten Forschungsaufgabe sollen Nachwuchsgruppen zugleich - über das Forschungsergebnis im engeren Sinn hinaus - die Kultur interdisziplinären wissenschaftlichen Arbeitens pflegen und entwickeln. Insgesamt soll eine weitere Öffnung der Universitäten für inter- und transdisziplinäre Forschungsansätze erreicht werden.
Die Laufzeit der Projekte beträgt fünf Jahre. Damit soll den besonderen Herausforderungen inter- und transdisziplinärer Forschung Rechnung getragen werden und zugleich die akademische Qualifizierung der Nachwuchswissenschaftler/-innen sichergestellt werden.

Von 2002 - 2014 wurden in zwei Förderphasen mit insgesamt 37 Mio. € Fördervolumen 21 Nachwuchsgruppen gefördert. Über 100 junge Wissenschaftlerinnen und Wissenschaftler konnten sich weiterqualifizieren. Die umfassenden Erfahrungen wurden mit Fragebögen erhoben und in verschiedenen Workshops analysiert. Basierend auf den gewonnenen Erkenntnissen (Bilanzierungsbericht) wurde die Nachwuchsgruppenförderung in der Sozial-ökologischen Forschung weiterentwickelt und in 2015 mit regelmäßigen Abgabeterminen neu ausgeschrieben.

Die abgeschlossenen Projekte der zweiten Förderphase finden sich hier.

Die sieben seit 2016 laufenden Nachwuchsgruppen (Förderphase 3) beschäftigen sich mit den Themen systemische Risiken in Bezug auf Kunststoffe, Saatgutproduktion, Obsoleszenz, Migration, Planung, Mobilität sowie Rebound-Effekte.

Für Förderphase 4 wurden fünf Projekte zur Förderung ausgewählt, die in 2017 bzw. 2018 gestartet sind bzw. starten sollen.

Die nächste Frist zur Skizzeneinreichung ist der 27.04.2018.

Im Folgenden werden die aktuell laufenden Nachwuchsgruppen jeweils kurz dargestellt.
